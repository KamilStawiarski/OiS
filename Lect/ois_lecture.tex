% Styl Politechniki Gdańskiej dedykowany do beamer'a, 
% Kompilowany przy pomocy pdflatex lub lualatex, z biblatex'em na backendzie biber'a
% W przyszłości będzie aktualizowany pod standard WCAG, PDF2.0/UA-2, 
% Szczegóły: https://github.com/latex3/tagging-project
% Aktualnie, częściowym obejściem problemu jest użycie pakietu accsupp 
\DocumentMetadata{
  lang        = pl,
  pdfversion  = 2.0,
  pdfstandard = ua-2,
  pdfstandard = a-4f, %or a-4
  testphase   = latest
}
\tagpdfsetup{
 math/mathml/structelem,
 math/tex/AF=false,
 math/mathml/AF=false,
 math/mathml/sources=true
}

\documentclass[9pt, aspectratio=169]{beamer}
%~~~~~~~~~~~~~~~~~~~~~~~~~~~~~~~~~~~~~~~~~~~~~~~~~~~~~~~~~~~~~~~~~~~~~~~~~~~~~~
% Kompilujemy przy użyciu LuaLaTex (czcionka prawie jak Arial), zaś preferowany manager bibliografii to biblatex 
%~~~~~~~~~~~~~~~~~~~~~~~~~~~~~~~~~~~~~~~~~~~~~~~~~~~~~~~~~~~~~~~~~~~~~~~~~~~~~~
\usetheme[faculty=weti, color=standard, copyrights=ccby, qr=./assets/pg_beamer_theme_qr.png, i5=true]{pg}
% Opcja faculty -- możliwe wydziały to: warch, wch, weia, weti, wftims, wilis, wimio, wzie
% Możliwe opcje kolorystyczne w ustawieniach \setbeamertemplate{frametitle}[]: standard, rojo, amartant, blend
% Opcja color wskazuje na użycie w całej prezentacji jednego koloru do momentu jego wymuszonej zmiany 
% Opcja copyrights -- pozwala na wybór prawa autorskiego: ccby, cc0, ccbysa, none
% Opcja qr wstawia qrcode na ostatnim slajdzie (można go wygenerować np. za pomocą chatagpt), który powinien odnosić się do linku do zajęć
% Opcja i5 ustawiona na true wymusza dodanie znaków projektu Inżynier 5.0

%~~~~~~~~~~~~~~~~~~~~~~~~~~~~~~~~~~~~~~~~~~~~~~~~~~~~~~~~~~~~~~~~~~~~~~~~~~~~~~
% Setup biblatex'a z backend'em biber:
% Aby skompilować bibliografię należy wywołać lualatex %; biber %; lualatex % * 2
% przykładowy config to VSCodium:
% "latex-workshop.latex.tools": [
% {"name": "lualatex",
%  "command": "lualatex",
%  "args": [
%  		"-synctex=1",
%  		"-interaction=nonstopmode",
%  		"-file-line-error",
%   	"-pdf",
%		"%DOC%"
%			]},
% {"name": "biber",
%  "command": "biber",
%  "args": ["%DOC%"]}]
%  "latex-workshop.latex.recipes": [		
%  {"name": "lualatex->biber->lualatex",
%	"tools": ["lualatex", "biber", "lualatex", "lualatex"]}]
%~~~~~~~~~~~~~~~~~~~~~~~~~~~~~~~~~~~~~~~~~~~~~~~~~~~~~~~~~~~~~~~~~~~~~~~~~~~~~~
% Wymagany setup biber'a
%~~~~~~~~~~~~~~~~~~~~~~~~~~~~~~~~~~~~~~~~~~~~~~~~~~~~~~~~~~~~~~~~~~~~~~~~~~~~~~
\usepackage[backend=biber,
	style=ieee,
	sortlocale=pl_PL,
	natbib=true,
	url=true, 
	doi=true,
	eprint=false]{biblatex}
\setbeamertemplate{bibliography item}{\insertbiblabel}
%~~~~~~~~~~~~~~~~~~~~~~~~~~~~~~~~~~~~~~~~~~~~~~~~~~~~~~~~~~~~~~~~~~~~~~~~~~~~~~
% Wskazanie pliku z bibliografią:
\addbibresource{biblio.bib}

%~~~~~~~~~~~~~~~~~~~~~~~~~~~~~~~~~~~~~~~~~~~~~~~~~~~~~~~~~~~~~~~~~~~~~~~~~~~~~~
% Fix dotyczący \emph w bibliografii
%~~~~~~~~~~~~~~~~~~~~~~~~~~~~~~~~~~~~~~~~~~~~~~~~~~~~~~~~~~~~~~~~~~~~~~~~~~~~~~
\makeatletter
\renewrobustcmd*{\mkbibemph}{\mkbibitalic}
\protected\long\def\blx@imc@mkbibemph#1{\blx@imc@mkbibitalic{#1}}

\addto\captionspolish{%
  \renewcommand{\proofname}{Dowód}
}
\makeatother
\uselanguage{Polish}
%~~~~~~~~~~~~~~~~~~~~~~~~~~~~~~~~~~~~~~~~~~~~~~~~~~~~~~~~~~~~~~~~~~~~~~~~~~~~~~
% Setup href - do wypełnienia przez autora
%~~~~~~~~~~~~~~~~~~~~~~~~~~~~~~~~~~~~~~~~~~~~~~~~~~~~~~~~~~~~~~~~~~~~~~~~~~~~~~
\hypersetup{
	pdftitle={Tytuł prezentacji},
	pdfauthor={Autor prezentacji},
	pdfsubject={Temat prezentacji},
	pdfkeywords={słowa kluczowe}, 
	unicode=true,
	pdfencoding=auto,
	pdflang=pl-PL,
	colorlinks=true,
	linkcolor=rojo,
	filecolor=rojo,
	urlcolor=amarant
}
%~~~~~~~~~~~~~~~~~~~~~~~~~~~~~~~~~~~~~~~~~~~~~~~~~~~~~~~~~~~~~~~~~~~~~~~~~~~~~~
% Dane umieszczone w slajdzie tytułowym i stopce
%~~~~~~~~~~~~~~~~~~~~~~~~~~~~~~~~~~~~~~~~~~~~~~~~~~~~~~~~~~~~~~~~~~~~~~~~~~~~~~
\title[Tytuł w stopce]{Obwody i sygnały}
%\subtitle{Podtytuł}
\author[Autor w stopce]{dr inż. Kamil Stawiarski}
\institute{Politechnika Gdańska}
\date[Data w stopce]{\today}
%~~~~~~~~~~~~~~~~~~~~~~~~~~~~~~~~~~~~~~~~~~~~~~~~~~~~~~~~~~~~~~~~~~~~~~~~~~~~~~

\begin{document}

% Generuje stronę tytułową 
\titlepage

% Generuje spis treści 
\begin{frame}[allowframebreaks]{Spis treści}
	\tagtool{sec-add-grouping=false}
	\tableofcontents
	\tagtool{sec-add-grouping}
\end{frame}

\sect{Sprawy organizacyjne}

\begin{frame}{Sprawy organizacyjne}
	\begin{block}{Konsultacje i kontakt}
		\begin{itemize}
			\item[\bt] Osoba odpowiedzialna za przedmiot: dr inż. K. Stawiarski
			\item[\bt] Mail: kamil.stawiarski@pg.edu.pl
			\item[\bt] Konsultacje: wtorki, 7:15 do 8:00 EA550
			\item[\bt] Preferowane konsultacje mailowe bądź online po wcześniejszym umówieniu
		\end{itemize}
	\end{block}
\end{frame}

\begin{frame}{Organizacja przedmiotu}
	\begin{itemize}
		\item[\bt] Wykład: 30 godzin (15x 90 min.)
		\item[\bt] Ćwiczenia: 30 godzin (30x 45 min.)
		\item[\bt] Wykład podzielony jest na 3 części:
		\begin{itemize}
			\item[\bt] Część I - stałoprądowa (6 wykładów)
			\item[\bt] Część II - zmiennoprądowa (5 wykładów)
			\item[\bt] Część III - sygnałowa (2 wykłady)
		\end{itemize}
		\item[\bt] Każda z części wykładu kończy się zaliczeniem.
	\end{itemize}
\end{frame}

\begin{frame}{Literatura i narzędzia}
	\begin{block}{Książki}
		\begin{itemize}
			\item[\bt] \textit{Podstawy teorii obwodów, tom 1, 2} Jerzy Osiowski, Jerzy Szabatin, Wydawnictwo Naukowe PWN
			\item[\bt] \textit{Engineering Circuit Analysis} J.D. Irwin, R.M. Nelms, Wiley
		\end{itemize}
	\end{block}
	\begin{block}{Symulatory}
		\begin{itemize}
			\item[\bt] \textit{Falstad} \url{https://www.falstad.com/circuit/circuitjs.html}
			\item[\bt] \textit{Xyce} \url{https://xyce.sandia.gov/}
			\item[\bt] \textit{Octave} \url{}
		\end{itemize}
	\end{block}
\end{frame}

\begin{frame}{Kalkulator}
	\begin{block}{Kalkulator}
		W trakcie zajęć oraz podczas kolokwiów i egzaminów bardzo przydatny będzie kalkulator naukowy. Bez jego użycia rozwiązanie zadań w wyznaczonym czasie może być praktycznie niemożliwe.
		Najlepiej sprawdzają się modele firmy Casio, zwłaszcza Casio FX-991CEX, który umożliwia wykonywanie obliczeń na liczbach zespolonych, macierzach, a także posiada wiele zaawansowanych funkcji przydatnych w teorii obwodów.
		Alternatywnie można rozważyć również inne modele, takie jak:
		\begin{itemize}
			\item[\bt] Casio FX-570EX
			\item[\bt] Casio FX-991EX
			\item[\bt] Sharp EL-W506T
			\item[\bt] Texas Instruments TI-36X Pro
		\end{itemize}
		Ważne, by wybrany kalkulator miał funkcje operacji na liczbach zespolonych oraz macierzach — tylko wtedy będzie w pełni funkcjonalny na zajęciach i podczas egzaminów.
	\end{block}
\end{frame}

\begin{frame}{Zasady zaliczenia}
	\begin{itemize}
		\item[\bt] Do zaliczenia są 2 kolokwia (część stałoprądowa, część zmiennoprądowa) i egzamin (część sygnałowa)
		\item[\bt] Aby zdać należy uzyskać:
		\begin{itemize}
			\item[\bt] Średni wynik z trzech części co najmniej 50\%
			\item[\bt] Co najmniej 40\% z każdej części
		\end{itemize}
	\end{itemize}
	\begin{block}{Oceny na podstawie średniej}
		\begin{itemize}
			\item[\bt] (50\%; 60\%) - 3
			\item[\bt] $<$60\%; 70\%) - 3,5
			\item[\bt] $<$70\%; 80\%) - 4
			\item[\bt] $<$80\%; 90\%) - 4,5
			\item[\bt] $<$90\%; 100\%$>$ - 5
		\end{itemize}
	\end{block}
\end{frame}

\begin{frame}{Zasady zaliczenia - przykład}
		\begin{table}
		\centering
		\tagpdfsetup{table/header-rows={1}}
		\caption{Przykładowe wyniki}
		\begin{tabular}{|c|c|c|c|c|c|}
			\hline
			K1 [\%] & K2 [\%] & E [\%] & Śr. $>$50\% ? & Każda $>$40\%? & Zdane? \\ \hline
			10      & 10      & 40     & Nie (20\%)  & Nie          & Nie    \\ \hline
			80      & 30      & 70     & Tak (70\%)  & Nie          & Nie    \\ \hline
			45      & 45      & 45     & Nie (45\%)  & Tak          & Nie    \\ \hline
			60      & 70      & 80     & Tak (70\%)  & Tak          & Tak    \\ \hline
		\end{tabular}
	\end{table}
\end{frame}

\begin{frame}{Terminy zaliczeń i poprawy}
	\begin{itemize}
		\item[\bt] Kolokwium 1 (część stałoprądowa) - w trakcie ćwiczeń w okolicy połowy semestru
		\item[\bt] Kolokwium 2 (część zmiennoprądowa) - po zajęciach, w ustalonym terminie, okolica końca semestru
		\item[\bt] Egzamin - w sesji
		\item[\bt] W trakcie egzaminu, oraz kolejnych terminów zaliczeń będzie można podejść do poprawy dowolnej części.
		\item[\bt] Wynik uzyskany z kolejnym terminie nadpisuje poprzednio uzyskany.
	\end{itemize}
\end{frame}





\begin{frame}{Poprawa - przykład}
	\begin{itemize}
		\item[\bt] Przyjmijmy, że na pierwsze kolokwium przeznaczony jest czas T1, na drugie T2 i na egzamin TE.
		\item[\bt] Osoba przychodząc na poprawę dostaję zestaw 3 arkuszy z 3 częściami
		\item[\bt] Przez pierwsze 10 minut można zapoznać się z zadaniami i oddać pusty arkusz, bez rozwiązywania zadań.
		\item[\bt] Po 10 min każda osoba wychodząca z sali musi oddać co najmniej 1 arkusz.
		\item[\bt] Osoby poprawiająca tylko kolokwium 1 oddają rozwiązania po czasie T1
		\item[\bt] Osoby poprawiająca tylko kolokwium 2 oddają rozwiązania po czasie T2
		\item[\bt] Osoby poprawiająca tylko egzamin oddają rozwiązania po czasie TE
		\item[\bt] Osoby poprawiająca kolokwium 1 i 2 oddają rozwiązania po czasie T1+T2
		\item[\bt] Osoby poprawiająca kolokwium 1 i egzamin oddają rozwiązania po czasie T1+TE
		\item[\bt] Osoby poprawiająca kolokwium 2 i egzamin oddają rozwiązania po czasie T2+TE
		\item[\bt] Po czasie T1+T2+TE egzamin się kończy i wszyscy do nadal piszący oddają arkusze z trzema częściami
	\end{itemize}
\end{frame}



\sect{Wykład 1}

\begin{frame}{Wielkości fizyczne}
	\begin{itemize}
		\item[\bt] Ładunek
		\item[\bt] Natężenie
		\item[\bt] Napięcie
	\end{itemize}
\end{frame}

\begin{frame}{Prawo Ohma}
	\begin{itemize}
		\item[\bt] Rezystancja Om/Siemens
	\end{itemize}
\end{frame}

\begin{frame}{Energia}
	\begin{itemize}
		\item[\bt] Dżul/Wat
	\end{itemize}
\end{frame}

\begin{frame}{Definicja obwodu elektrycznego}
	Placeholder
\end{frame}

\begin{frame}{Przedrostki + notacja wykładnicza}
	Placeholder
\end{frame}

\begin{frame}{Prądowe prawo Kirchoffa (PPK)}
	Placeholder
\end{frame}

\begin{frame}{Napięciowe prawi Kirchofa (NPK)}
	Placeholder
\end{frame}

\begin{frame}{Zasada zachowania energii (zasada Tellgena)}
	Placeholder
\end{frame}

\begin{frame}{Elementy w obwodzie}
	\begin{itemize}
		\item[\bt] Rezystory
		\item[\bt] Źródła niezależne
		\item[\bt] Żródła zależne
		\item[\bt] Wzmacniacze operacyjne
    \end{itemize}
\end{frame}

\begin{frame}{Definicje SLS}
	Placeholder
\end{frame}

\begin{frame}{Strzałkowanie w obwodzie}
	Placeholder
\end{frame}

\begin{frame}{Omówienie źródła napięciowego}
	Placeholder
\end{frame} 

\begin{frame}{Omówienie źródła prądowego}
	Placeholder
\end{frame} 

\sect{Wykład 2}

\begin{frame}{Łączenie rezystorów}
	Placeholder
\end{frame} 

\begin{frame}{Źródło Thevenina}
	Placeholder
\end{frame} 

\begin{frame}{Źródło Nortona}
	Placeholder
\end{frame} 

\begin{frame}{Połączenie G-T}
	Placeholder
\end{frame} 

\sect{Wykład 3}
\begin{frame}{Bilans mocy}
	Placeholder
\end{frame} 

\begin{frame}{Optymalny pobór mocy}
	Placeholder
\end{frame} 

\sect{Wykład 4}

\begin{frame}{Zasada superpozycji}
	Placeholder
\end{frame} 

\begin{frame}{Metoda zamiany źródeł}
	Placeholder
\end{frame} 

\sect{Wykład 5}
\begin{frame}{Równania oczkowe}
	Placeholder
\end{frame} 

\begin{frame}{Rówania węzłowe}
	Placeholder
\end{frame} 

\begin{frame}{Ogólna metoda rozwiązywania schematów}
	Placeholder
\end{frame} 

\sect{Wykład 6}
\begin{frame}{Źródła sterowane (4 sztuki)}
	Placeholder
\end{frame} 

\begin{frame}{Wzmacniacz operacyjny}
	Placeholder
\end{frame} 

\sect{Wykład 7}
\begin{frame}{Metoda prądów oczkowych}
	Placeholder
\end{frame} 

\begin{frame}{Metoda potencjałów węzłowych}
	Placeholder
\end{frame} 
 
 
 
\include{src/03_AC.tex}
\include{src/04_Signal.tex}

\sect{Inżynier 5.0}
\begin{frame}{Program Inżynier 5.0}
	\begin{block}{Wymagania}
		\begin{itemize}
			\item[\bt] Prezentacje w jednym stylu 
			\item[\bt] Wykład -- minimum 15 slajdów na każdą godzinę zmian
			\item[\bt] Wprowadzenia do laboratoriów / ćwiczeń / seminariów -- min 5 slajdów na temat
			\item[\bt] Wprowadzenie do projektów -- min 15 slajdów
			\item[\bt] Na slajdach brak tekstu ciągłego			
			\item[\bt] Treści / zadania do ćwiczeń, laboratoriów, seminarium w formatce A4 (Word / LaTeX)
			\item[\bt] Wymagania dotyczące dostępności -- WCAG 2.x oraz PDF/UA (tagowany pdf)
			\item[\bt] Narzędzie do weryfikacji tagów / kontrastów \url{https://pdfa.org/product/pac/}		
			\item[\bt] Spełnione prawa autorskie CC-BY (grafiki własne lub na licencjach otwartych -- np. CC lub MiT)
		\end{itemize}
	\end{block}
\end{frame}


% Bibliografia
\sect{Bibliografia}
\begin{frame}[allowframebreaks]{Bibliografia}
	\nocite{*}
	% W opcji czystego bibtex'a:
	%\bibliographystyle{plain}
	%\bibliography{biblio}
	%
	% W opcji biblatex:
	\renewcommand*{\bibfont}{\normalfont\small}
	\printbibliography
\end{frame}

% Strona końcowa tematu PG
\endpage

\end{document}