\sect{Sprawy organizacyjne}

\begin{frame}{Sprawy organizacyjne}
	\begin{block}{Konsultacje i kontakt}
		\begin{itemize}
			\item[\bt] Osoba odpowiedzialna za przedmiot: dr inż. K. Stawiarski
			\item[\bt] Mail: kamil.stawiarski@pg.edu.pl
			\item[\bt] Konsultacje: wtorki, 7:15 do 8:00 EA550
			\item[\bt] Preferowane konsultacje mailowe bądź online po wcześniejszym umówieniu
		\end{itemize}
	\end{block}
\end{frame}

\begin{frame}{Organizacja przedmiotu}
	\begin{itemize}
		\item[\bt] Wykład: 30 godzin (15x 90 min.)
		\item[\bt] Ćwiczenia: 30 godzin (30x 45 min.)
		\item[\bt] Wykład podzielony jest na 3 części:
		\begin{itemize}
			\item[\bt] Część I - stałoprądowa (6 wykładów)
			\item[\bt] Część II - zmiennoprądowa (5 wykładów)
			\item[\bt] Część III - sygnałowa (2 wykłady)
		\end{itemize}
		\item[\bt] Każda z części wykładu kończy się zaliczeniem.
	\end{itemize}
\end{frame}

\begin{frame}{Literatura i narzędzia}
	\begin{block}{Książki}
		\begin{itemize}
			\item[\bt] \textit{Podstawy teorii obwodów, tom 1, 2} Jerzy Osiowski, Jerzy Szabatin, Wydawnictwo Naukowe PWN
			\item[\bt] \textit{Engineering Circuit Analysis} J.D. Irwin, R.M. Nelms, Wiley
		\end{itemize}
	\end{block}
	\begin{block}{Symulatory}
		\begin{itemize}
			\item[\bt] \textit{Falstad} \url{https://www.falstad.com/circuit/circuitjs.html}
			\item[\bt] \textit{Xyce} \url{https://xyce.sandia.gov/}
			\item[\bt] \textit{Octave} \url{}
		\end{itemize}
	\end{block}
\end{frame}

\begin{frame}{Kalkulator}
	\begin{block}{Kalkulator}
		W trakcie zajęć oraz podczas kolokwiów i egzaminów bardzo przydatny będzie kalkulator naukowy. Bez jego użycia rozwiązanie zadań w wyznaczonym czasie może być praktycznie niemożliwe.
		Najlepiej sprawdzają się modele firmy Casio, zwłaszcza Casio FX-991CEX, który umożliwia wykonywanie obliczeń na liczbach zespolonych, macierzach, a także posiada wiele zaawansowanych funkcji przydatnych w teorii obwodów.
		Alternatywnie można rozważyć również inne modele, takie jak:
		\begin{itemize}
			\item[\bt] Casio FX-570EX
			\item[\bt] Casio FX-991EX
			\item[\bt] Sharp EL-W506T
			\item[\bt] Texas Instruments TI-36X Pro
		\end{itemize}
		Ważne, by wybrany kalkulator miał funkcje operacji na liczbach zespolonych oraz macierzach — tylko wtedy będzie w pełni funkcjonalny na zajęciach i podczas egzaminów.
	\end{block}
\end{frame}

\begin{frame}{Zasady zaliczenia}
	\begin{itemize}
		\item[\bt] Do zaliczenia są 2 kolokwia (część stałoprądowa, część zmiennoprądowa) i egzamin (część sygnałowa)
		\item[\bt] Aby zdać należy uzyskać:
		\begin{itemize}
			\item[\bt] Średni wynik z trzech części co najmniej 50\%
			\item[\bt] Co najmniej 40\% z każdej części
		\end{itemize}
	\end{itemize}
	\begin{block}{Oceny na podstawie średniej}
		\begin{itemize}
			\item[\bt] (50\%; 60\%) - 3
			\item[\bt] $<$60\%; 70\%) - 3,5
			\item[\bt] $<$70\%; 80\%) - 4
			\item[\bt] $<$80\%; 90\%) - 4,5
			\item[\bt] $<$90\%; 100\%$>$ - 5
		\end{itemize}
	\end{block}
\end{frame}

\begin{frame}{Zasady zaliczenia - przykład}
		\begin{table}
		\centering
		\tagpdfsetup{table/header-rows={1}}
		\caption{Przykładowe wyniki}
		\begin{tabular}{|c|c|c|c|c|c|}
			\hline
			K1 [\%] & K2 [\%] & E [\%] & Śr. $>$50\% ? & Każda $>$40\%? & Zdane? \\ \hline
			10      & 10      & 40     & Nie (20\%)  & Nie          & Nie    \\ \hline
			80      & 30      & 70     & Tak (70\%)  & Nie          & Nie    \\ \hline
			45      & 45      & 45     & Nie (45\%)  & Tak          & Nie    \\ \hline
			60      & 70      & 80     & Tak (70\%)  & Tak          & Tak    \\ \hline
		\end{tabular}
	\end{table}
\end{frame}

\begin{frame}{Terminy zaliczeń i poprawy}
	\begin{itemize}
		\item[\bt] Kolokwium 1 (część stałoprądowa) - w trakcie ćwiczeń w okolicy połowy semestru
		\item[\bt] Kolokwium 2 (część zmiennoprądowa) - po zajęciach, w ustalonym terminie, okolica końca semestru
		\item[\bt] Egzamin - w sesji
		\item[\bt] W trakcie egzaminu, oraz kolejnych terminów zaliczeń będzie można podejść do poprawy dowolnej części.
		\item[\bt] Wynik uzyskany z kolejnym terminie nadpisuje poprzednio uzyskany.
	\end{itemize}
\end{frame}





\begin{frame}{Poprawa - przykład}
	\begin{itemize}
		\item[\bt] Przyjmijmy, że na pierwsze kolokwium przeznaczony jest czas T1, na drugie T2 i na egzamin TE.
		\item[\bt] Osoba przychodząc na poprawę dostaję zestaw 3 arkuszy z 3 częściami
		\item[\bt] Przez pierwsze 10 minut można zapoznać się z zadaniami i oddać pusty arkusz, bez rozwiązywania zadań.
		\item[\bt] Po 10 min każda osoba wychodząca z sali musi oddać co najmniej 1 arkusz.
		\item[\bt] Osoby poprawiająca tylko kolokwium 1 oddają rozwiązania po czasie T1
		\item[\bt] Osoby poprawiająca tylko kolokwium 2 oddają rozwiązania po czasie T2
		\item[\bt] Osoby poprawiająca tylko egzamin oddają rozwiązania po czasie TE
		\item[\bt] Osoby poprawiająca kolokwium 1 i 2 oddają rozwiązania po czasie T1+T2
		\item[\bt] Osoby poprawiająca kolokwium 1 i egzamin oddają rozwiązania po czasie T1+TE
		\item[\bt] Osoby poprawiająca kolokwium 2 i egzamin oddają rozwiązania po czasie T2+TE
		\item[\bt] Po czasie T1+T2+TE egzamin się kończy i wszyscy do nadal piszący oddają arkusze z trzema częściami
	\end{itemize}
\end{frame}


