\sect{Wykład 14}

\begin{frame}{Charakterystyki częstotliwościowe}
Charakterystyka Bodego - charakterystyka częstotliwościowa logarytmiczna - charakterystyka wzmocnienia (charakterystyka amplitudowa) oraz przesunięcie fazy (charakterystyka fazowa) w funkcji częstotliwości.
	\begin{figure}
		\includegraphics[scale=0.2, alt={Na obrazku widoczny jest jeden wykres przedstawiający przebiegi częstotliwościowe kilku typów głośników, wszystkie odniesione do średniego poziomu ciśnienia akustycznego.
		Oś pozioma pokazuje częstotliwość w hercach w skali logarytmicznej, od około dziesięciu herców po lewej stronie do ponad dziesięciu tysięcy herców po prawej. Oś pionowa to poziom ciśnienia akustycznego w decybelach, od około minus sto na dole do około plus dwudziestu na górze.
		Na wykresie są cztery krzywe o różnych odcieniach szarości, podpisane u góry jako studio monitor, soundbar, flat panel oraz smart speaker. Krzywe te pokazują, jak każdy z tych głośników odtwarza pasmo: jedne mają stosunkowo płaską charakterystykę w szerokim zakresie częstotliwości, inne wyraźnie opadają w basie lub w najwyższych częstotliwościach, co widać jako spadki lub falowania linii.​}]{./imags/L14/Frequency-response-on-0-deg-axis-of-all-speakers-referenced-to-their-SPL-mean-and-shifted.png}	
		\caption{Przykład charakterystyki częstotliwościowej.}
	\end{figure}	
\end{frame}

\begin{frame}{Charakterystyki częstotliwościowe}
	\begin{itemize}
		\item[\bt] Na ogół os pozioma jest logarytmiczną skalą częstotliwości.
		\item[\bt] W charakterystyce amplitudowej - oś pionowa jest wzmocnieniem w dB
		\item[\bt] W charakterystyce fazowej - oś pionowa jest liniowa od $-180[^\circ]$ do $180[^\circ]$
	\end{itemize}
	\begin{equation*}
		K_{[dB]} = 20 log_{10} \left( \frac{A}{A_0} \right)
	\end{equation*}
\end{frame}

\begin{frame}{Filtry pasywne}
	Układy z elementami RLC w których stosunek zespolonych amplitud napięcia wyjściowego do wejściowego (transmitancja widmowa, $H(j\omega)$ jest zależny od częstotliwości.
	\begin{figure}
		\includegraphics[scale=0.2, alt={Na obrazku znajdują się cztery schematyczne wykresy idealnych charakterystyk filtrów, rozmieszczone w prostokąt: dwa u góry i dwa u dołu.
		W lewym górnym rogu czerwony wykres dolnoprzepustowy ma postać prostokąta: od lewej strony jest na wysokim poziomie, a w pewnym punkcie gwałtownie spada do zera i dalej pozostaje na zerze, co oznacza idealne przepuszczanie wszystkich częstotliwości poniżej granicy i całkowite tłumienie powyżej. W prawym górnym rogu fioletowy wykres górnoprzepustowy jest odwrotny: od lewej jest na poziomie zera, w pewnym miejscu skokowo rośnie do maksymalnej wartości i dalej utrzymuje się na tym poziomie, pokazując przepuszczanie wyłącznie częstotliwości powyżej częstotliwości odcięcia.
		W lewym dolnym rogu zielony wykres środkowoprzepustowy tworzy pionowo wznoszący się bok, następnie poziomy odcinek na wysokim poziomie i pionowy spadek, jak prostokąt "stojący" w środku osi poziomej; oznacza to przepuszczanie tylko pasma częstotliwości pośrednich, przy pełnym tłumieniu niskich i wysokich. W prawym dolnym rogu szary wykres środkowozaporowy ma odwrotny kształt: poziomą linię na wysokim poziomie, w środku prostokątny "dołek" spadający do zera i znów powrót do wysokiego poziomu, co odpowiada idealnemu tłumieniu pasma środkowego przy przepuszczaniu częstotliwości poniżej i powyżej tego zakresu.​}]{./imags/L14/f_ideal.png}	
		\caption{Idealna odpowiedź częstotliwościowa.}
	\end{figure}	
\end{frame}

\begin{frame}{Filtry pasywne}
	\begin{itemize}
		\item[\bt] Pasmo przenoszenia - zakres częstotliwości w których moduł amplitudy jest większy niż $-3dB$: $\omega: |H(j \omega)| > \frac{1}{\sqrt{2}}$
		\item[\bt] Pasmo zaporowe -  zakres częstotliwości w których moduł amplitudy jest mniejszy niż $-3dB$: $\omega: |H(j \omega)| < \frac{1}{\sqrt{2}}$
	\end{itemize}
	\begin{figure}
		\includegraphics[scale=0.2, alt={Na obrazku znajdują się cztery małe wykresy charakterystyk filtrów, ułożone w prostokąt: dwa u góry, dwa u dołu.
		W lewym górnym rogu jest czerwony wykres dolnoprzepustowy: linia zaczyna się wysoko z lewej strony i przy pewnej częstotliwości opada w dół, pokazując, że niskie częstotliwości są przepuszczane, a wyższe tłumione. W prawym górnym rogu znajduje się niebieski wykres górnoprzepustowy: z lewej strony jest niski, a następnie gwałtownie rośnie i dalej pozostaje wysoki, co oznacza tłumienie niskich i przepuszczanie wysokich częstotliwości.
		W lewym dolnym rogu jest zielony wykres środkowoprzepustowy: krzywa rośnie z niskiej wartości, tworzy wypukły "garb" w środku i potem znowu spada, co oznacza przepuszczanie tylko pasma częstotliwości pośrednich. W prawym dolnym rogu szary wykres środkowozaporowy ma kształt odwrotny: od lewej i prawej strony jest wysoki, a w środku tworzy głęboki dołek, pokazując tłumienie częstotliwości środkowych i przepuszczanie niskich oraz wysokich..​}]{./imags/L14/f_real.png}
		\caption{Przybliżona odpowiedź częstotliwościowa.}
	\end{figure}
\end{frame}


\begin{frame}{Dobroć filtra}
	\begin{itemize}
		\item[\bt] Dobroć filtra - jako stosunek częstotliwości środkowej filtru do szerokości jego pasma: $Q=\frac{f_{g,3dB} - f_{d,3dB}}{f_0}$. 
		Odnosi się do filtrów pasmowo - przepustowych.
		\item[\bt] Rząd filtru - stopień wielomianu transmitancji operatorowej filtra - definiuje nachylenie zbocza filtra góro i dolnoprzepustowego.
	\end{itemize}
\end{frame}

\begin{frame}{Dobroć filtra}
	\begin{figure}
		\includegraphics[scale=0.4, alt={Na obrazku jest pojedynczy wykres pokazujący, jak zmienia się wzmocnienie filtru w zależności od częstotliwości, dla trzech różnych wartości współczynnika dobroci Q.
		Oś pozioma to częstotliwość w hercach, od około jednego kiloherca do dwóch przecinek czterech kiloherca. Oś pionowa to amplituda w decybelach, od zera na górze do około minus dwudziestu na dole.
		Na wykresie znajdują się trzy krzywe. Zielona, szeroka, łagodnie zaokrąglona na górze, jest opisana jako Q równe 1 i ma maksimum w okolicach częstotliwości środkowej, ale zmienia się powoli. Niebieska krzywa o średniej szerokości jest podpisana Q równe 10 i jest znacznie bardziej stroma, z wyraźnym szpicem w środku. Żółto-brązowa krzywa jest bardzo wąska i wysoka, podpisana Q równe 100, tworzy prawie pionowy szczyt w pobliżu częstotliwości środkowej.
		W środku wykresu, przy częstotliwości około jeden przecinek sześć kiloherca, znajduje się pionowa linia i opis szesnaście herców z poziomą strzałką nad nią oznaczoną jako sto pięćdziesiąt dziewięć herców, co wskazuje szerokość pasma wokół częstotliwości środkowej. Obraz ilustruje, że im większa dobroć Q, tym węższe pasmo przepuszczanych częstotliwości i bardziej strome zbocza charakterystyki.​}]{./imags/L14/Q-Factor.png}	
		\caption{Przykład filtrów o różnej dobroci.}
	\end{figure}
\end{frame}

\begin{frame}{Analogowy filtr Czebyszewa}
	Właściwości:
	\begin{itemize}
		\item[\bt] Minimalizacja Pasma Przepustowego: Filtr Czebyszewa jest zaprojektowany tak, aby mieć bardzo ostre zbocza, pasmo przejściowe jest krótsze w porównaniu do innych filtrów, (np. Butterwortha). Jednak osiąga to kosztem zafalowań w paśmie przepustowym.
		\item[\bt] Rodzaje: Są dwa główne typy filtrów Czebyszewa:
		\begin{itemize}
			\item[\bt] Typu I: Charakteryzuje się falowaniem w paśmie przepustowym, podczas gdy pasmo zaporowe jest monotoniczne.
			\item[\bt] Typu II (Inverse Chebyshev): Charakteryzuje się falowaniem w paśmie zaporowym, a pasmo przepustowe jest monotoniczne.
		\end{itemize}
		\item[\bt] Zastosowanie: Filtry Czebyszewa są często używane w aplikacjach, gdzie istotne jest, aby szybko zmniejszyć poziom sygnałów poza pasmem przepustowym, na przykład w systemach komunikacyjnych i systemach przetwarzania sygnałów.
	\end{itemize}
\end{frame}

\begin{frame}{Analogowy filtr Butterwortha}
	Właściwości:
	\begin{itemize}
		\item[\bt] Monotoniczna Charakterystyka Amplitudowa: Filtr Butterwortha ma monotoniczną charakterystykę amplitudową zarówno w paśmie przepustowym, jak i zaporowym. Oznacza to, że nie występują żadne oscylacje..
		\item[\bt] Płaska Charakterystyka w Paśmie Przepustowym: Filtr Butterwortha jest zaprojektowany tak, aby mieć jak najbardziej płaską charakterystykę w paśmie przepustowym. Dzięki temu minimalizuje się zniekształcenia sygnału w paśmie przepustowym.
		\item[\bt] Stopniowe Zbocza: W porównaniu do filtrów Czebyszewa, filtr Butterwortha ma bardziej łagodne zbocza charakterystyki częstotliwościowej. Przejście z pasma przepustowego do pasma zaporowego jest bardziej stopniowe.
		\item[\bt] Stabilność: Filtry Butterwortha są stabilne i mają dobrą odpowiedź impulsową, co jest korzystne w wielu aplikacjach inżynieryjnych.
	\end{itemize}
\end{frame}

\begin{frame}{Filtry pasywne}
	\href{https://www.falstad.com/afilter/circuitjs.html?startCircuit=filt-hipass.txt}{Przykłady filtrów pasywnych}
\end{frame}

\begin{frame}{Filtry aktywne}
	Analogowe filtry aktywne wykorzystują wzmacniacze operacyjne (op-ampy) w połączeniu z rezystorami, kondensatorami i ewentualnie cewkami do realizacji różnych funkcji filtracji.
Te filtry są często bardziej wydajne i elastyczne niż pasywne filtry analogowe, które składają się tylko z rezystorów, kondensatorów i cewek.\\
	\textbf{Zalety Filtrów Aktywnych}
	\begin{itemize}
		\item[\bt] Wzmocnienie: Mogą wzmacniać sygnał, co jest niemożliwe w przypadku pasywnych filtrów.
		\item[\bt] Brak induktorów: Unikają użycia cewek, które mogą być duże, ciężkie i kosztowne oraz mają problemy z indukcyjnością pasożytniczą.
		\item[\bt] Precyzja i kontrola: Łatwiejsze do precyzyjnej regulacji charakterystyk filtracji dzięki dokładnym wartościom komponentów i możliwości modyfikacji układu.
	\end{itemize}
\end{frame}

\begin{frame}{Filtry aktywne}
	\href{https://www.falstad.com/afilter/circuitjs.html?startCircuit=filt-hipass.txt}{Przykłady filtrów aktywnych}
\end{frame}
