\sect{Wykład 5}

\begin{frame}{Równania węzłowe}
	\begin{block}{Równania węzłowe}
		Równania prądowe w obwodzie to równania, które zapisuje się na podstawie I prawa Kirchhoffa (prawa prądowego) w węzłach obwodu. 
		Określają one, jak prądy "rozchodzą się" w rozgałęzieniach, zgodnie z zasadą zachowania ładunku.
	\end{block}
\end{frame} 

\begin{frame}{Równania węzłowe}
	\begin{block}{Istota równań prądowych}
		I prawo Kirchhoffa mówi, że suma prądów wpływających do węzła jest równa sumie prądów z niego wypływających, czyli ładunek w węźle nie może się gromadzić ani znikać.\\
		W zapisie algebraicznym przyjmuje się znak "+" dla prądów wpływających i "-" dla wypływających, a równanie ma postać $\sum i=0 A$ dla danego węzła.
	\end{block}
\end{frame} 

\begin{frame}{Równania węzłowe}
	\begin{block}{Jak je się zapisuje}
		Wybiera się węzły obwodu (punkty, gdzie łączą się co najmniej trzy gałęzie) i dla każdego z nich zapisuje się równanie, uwzględniając wszystkie prądy, które do niego dochodzą i z niego wychodzą. \\
		Dla prostego węzła z prądami $i_1$, $i_2$ wpływającymi i $i_3$ wypływającym równanie prądowe będzie np. $i_1+i_2-i_3=0A$ lub równoważnie $i_1+i_2=i_3$.
	\end{block}
\end{frame} 

\begin{frame}{Równania węzłowe}
	\begin{block}{Do czego służą}
		Zestaw równań prądowych (wraz z równaniami elementów typu prawo Ohma i równaniami napięciowymi) pozwala obliczyć nieznane prądy w gałęziach obwodu.\\
		Używa się ich w analizie zarówno prostych, jak i złożonych obwodów, zwłaszcza tam, gdzie występuje wiele rozgałęzień i trzeba wyznaczyć "rozpływ" prądu w całej sieci.
	\end{block}
\end{frame}     

\begin{frame}{Równania oczkowe}
	\begin{block}{Równania oczkowe}
		Równania oczkowe w obwodzie to równania, które zapisuje się na podstawie II prawa Kirchhoffa (prawa napięciowego) w oczkach obwodu.
		Opisują one zależności między spadkami napięć i źródłami napięcia w zamkniętych pętlach, zgodnie z zasadą zachowania energii.
	\end{block}
\end{frame}

\begin{frame}{Równania oczkowe}
	\begin{block}{Istota równań oczkowych}
		II prawo Kirchhoffa mówi, że w każdym zamkniętym oczku suma przyrostów i spadków napięć jest równa zero.\\
		W zapisie algebraicznym, przechodząc wzdłuż oczka, przyjmuje się odpowiednie znaki dla spadków napięć i źródeł, a równanie ma postać $\sum u=0V$ dla danego oczka.
	\end{block}
\end{frame}

\begin{frame}{Równania oczkowe}
	\begin{block}{Jak je się zapisuje}
		Wybiera się niezależne oczka obwodu (najmniejsze zamknięte pętle) i dla każdego z nich zapisuje się równanie, sumując spadki napięć na elementach oraz uwzględniając źródła napięcia. \\
		Dla prostego oczka z rezystorami o spadkach $u_1$, $u_2$ i źródłem napięcia $e$ równanie oczkowe może mieć postać $-e+u_1+u_2=0V$ lub równoważnie $e=u_1+u_2$.
	\end{block}
\end{frame}

\begin{frame}{Równania oczkowe}
	\begin{block}{Do czego służą}
	Zestaw równań oczkowych (wraz z prawem Ohma i zależnościami elementów) pozwala obliczyć nieznane prądy oczkowe, a następnie prądy w poszczególnych gałęziach obwodu. \\
	Metoda równań oczkowych jest szczególnie przydatna w obwodach płaskich z wieloma gałęziami, ponieważ umożliwia zmniejszenie liczby niewiadomych w porównaniu z metodą równań węzłowych.
	\end{block}
\end{frame}

\begin{frame}{Równania węzłowe - przykład}
	\begin{columns}
		\begin{column}{0.4\textwidth}
			Po prawej stronie przedstawiono przykładowe zadania.
			Zaznaczone na nim są wartości rezystorów, napięcie źródła $e$, prąd źródła $j$.
			Będzie ono stanowiło bazę do dalszych rozważań.
		\end{column}
		\begin{column}{0.6\textwidth}
			\begin{figure}
				\includegraphics[scale=0.08, alt={Na rysunku jest prosty, poziomy obwód elektryczny z trzema rezystorami w górnej gałęzi, dwoma rezystorami pionowymi w dół oraz dwoma źródłami: napięcia po lewej i prądowym po prawej.
				Po lewej stronie, w dolnej części, znajduje się źródło napięcia opisane jako e równe pięć woltów, narysowane jako okrąg ze strzałką skierowaną w górę.
				Od górnego zacisku tego źródła biegnie poziomy przewód w prawo, w którym kolejno włączone są trzy rezystory: najpierw R cztery o wartości cztery omy, a dalej R jeden o wartości jeden om.
				Między R cztery a R jeden znajduje się węzeł, z którego pionowo w dół podłączony jest rezystor R dwa o wartości dwa omy, połączony z dolną prostą szyną odniesienia.
				Za rezystorem R jeden, bardziej w prawo, jest kolejny węzeł, z którego pionowo w dół podłączony jest rezystor R trzy o wartości trzy omy, również do tej samej dolnej szyny.
				Po prawej stronie dolnej gałęzi znajduje się źródło prądowe przedstawione jako okrąg ze strzałką w górę, opisane jako j równe jeden amper, włączone między prawym końcem dolnej szyny a prawym końcem górnej gałęzi obwodu.}]{./imags/L5/fig2a.png}	
				\caption{Przykładowe zadanie.}
			\end{figure}	
		\end{column}
	\end{columns}
\end{frame} 

\begin{frame}{Równania węzłowe - przykład}
	\begin{columns}
		\begin{column}{0.4\textwidth}
			Na schemacie po prawej usunięto oznaczenia wartości, dodano za to prądy na napięcia na elementach, węzły oraz oczka.
		\end{column}
		\begin{column}{0.6\textwidth}
			\begin{figure}
				\includegraphics[scale=0.08, alt={Na rysunku znajduje się ten sam obwód co poprzednio, ale uzupełniony o oznaczenia prądów, spadków napięć oraz trzy oczka oznaczone O jeden, O dwa i O trzy, a także o oznaczenia trzech węzłów W jeden, W dwa oraz dolnej szyny jako W trzy.
				Po lewej w dolnej gałęzi nadal jest źródło napięcia e, obok którego zaznaczono strzałką kierunek napięcia od dołu do góry.
				Na górze, w pierwszym rezystorze po lewej, oznaczonym wcześniej jako R cztery, jest strzałka prądu skierowana w prawo z opisem i e, obok zaznaczono też spadek napięcia u R cztery strzałką skierowaną w lewo.
				W punkcie między R cztery a R jeden znajduje się węzeł W jeden, z którego w dół biegnie rezystor R dwa z prądem i R dwa skierowanym w dół i ze strzałką napięcia u R dwa skierowaną w górę.
				W kolejnym rezystorze w górnej gałęzi, R jeden, zaznaczono prąd i R jeden skierowany w lewo oraz spadek napięcia u R jeden strzałką w prawo; węzeł po prawej stronie tego rezystora opisany jest jako W dwa.
				Z węzła W dwa w dół biegnie rezystor R trzy, przy którym strzałka prądu i R trzy skierowana jest w dół, a napięcie u R trzy zaznaczono strzałką w górę.
				Po prawej stronie dolnej gałęzi nadal znajduje się źródło prądowe z prądem j skierowanym w górę, a obok zaznaczono napięcie u j między górnym a dolnym zaciskiem źródła.
				Wewnątrz obwodu narysowano trzy pętle oczkowe: po lewej oczko O jeden obejmuje źródło napięcia, rezystor R cztery i rezystor R dwa; środkowe oczko O dwa obejmuje rezystory R jeden, R dwa i R trzy; prawe oczko O trzy obejmuje rezystor R trzy i źródło prądowe.
				Przy dolnej szynie zaznaczono opis W trzy, wskazując, że jest to wspólny węzeł odniesienia dla wszystkich gałęzi obwodu.}]{./imags/L5/fig2b.png}	
				\caption{Przykładowe zadanie.}
			\end{figure}	
		\end{column}
	\end{columns}
\end{frame} 

\begin{frame}{Równania węzłowe - przykład}
	\begin{columns}
		\begin{column}{0.4\textwidth}
			Na początku skupmy się na węzłach.
			Można zauważyć, że w obwodzie występują 4 węzły, jednak w rzeczywistości, dwa dolne węzły stanowią jeden węzeł (oznaczony jako $W_3$).
			Można sobie to wyobrazić jako utworzenia trójkąta z rezystorów $R_1$, $R_2$ i $R_3$.
			Wtedy będzie dobrze widoczne że na dole istnieje jeden węzeł.
		\end{column}
		\begin{column}{0.6\textwidth}
			\begin{figure}
				\includegraphics[scale=0.08, alt={Na rysunku znajduje się ten sam obwód co poprzednio, ale uzupełniony o oznaczenia prądów, spadków napięć oraz trzy oczka oznaczone O jeden, O dwa i O trzy, a także o oznaczenia trzech węzłów W jeden, W dwa oraz dolnej szyny jako W trzy.
				Po lewej w dolnej gałęzi nadal jest źródło napięcia e, obok którego zaznaczono strzałką kierunek napięcia od dołu do góry.
				Na górze, w pierwszym rezystorze po lewej, oznaczonym wcześniej jako R cztery, jest strzałka prądu skierowana w prawo z opisem i e, obok zaznaczono też spadek napięcia u R cztery strzałką skierowaną w lewo.
				W punkcie między R cztery a R jeden znajduje się węzeł W jeden, z którego w dół biegnie rezystor R dwa z prądem i R dwa skierowanym w dół i ze strzałką napięcia u R dwa skierowaną w górę.
				W kolejnym rezystorze w górnej gałęzi, R jeden, zaznaczono prąd i R jeden skierowany w lewo oraz spadek napięcia u R jeden strzałką w prawo; węzeł po prawej stronie tego rezystora opisany jest jako W dwa.
				Z węzła W dwa w dół biegnie rezystor R trzy, przy którym strzałka prądu i R trzy skierowana jest w dół, a napięcie u R trzy zaznaczono strzałką w górę.
				Po prawej stronie dolnej gałęzi nadal znajduje się źródło prądowe z prądem j skierowanym w górę, a obok zaznaczono napięcie u j między górnym a dolnym zaciskiem źródła.
				Wewnątrz obwodu narysowano trzy pętle oczkowe: po lewej oczko O jeden obejmuje źródło napięcia, rezystor R cztery i rezystor R dwa; środkowe oczko O dwa obejmuje rezystory R jeden, R dwa i R trzy; prawe oczko O trzy obejmuje rezystor R trzy i źródło prądowe.
				Przy dolnej szynie zaznaczono opis W trzy, wskazując, że jest to wspólny węzeł odniesienia dla wszystkich gałęzi obwodu.}]{./imags/L5/fig2b.png}	
				\caption{Przykładowe zadanie.}
			\end{figure}	
		\end{column}
	\end{columns}
\end{frame} 


\begin{frame}{Równania węzłowe - przykład}
	\begin{columns}
		\begin{column}{0.4\textwidth}
			Strzałki prądowe oznaczone zostały w sposób dowolny, warto jednak przyjąć, że są one zgodne z kierunkowością źródeł.
			Na rezystorach można oznaczać dowolnie.
			Istotne jest za to aby strzałki napięć na rezystorach były skierowane przeciwnie do strzałek przepływających przez nie prądów.
			W przypadku kiedy strzałka powinna być skierowana odwrotnie - w wyniku wyjdzie ujemny prąd i napięcie (to nie jest problemem, moc nadal wyjdzie dodatnia).
		\end{column}
		\begin{column}{0.6\textwidth}
			\begin{figure}
				\includegraphics[scale=0.08, alt={Na rysunku znajduje się ten sam obwód co poprzednio, ale uzupełniony o oznaczenia prądów, spadków napięć oraz trzy oczka oznaczone O jeden, O dwa i O trzy, a także o oznaczenia trzech węzłów W jeden, W dwa oraz dolnej szyny jako W trzy.
				Po lewej w dolnej gałęzi nadal jest źródło napięcia e, obok którego zaznaczono strzałką kierunek napięcia od dołu do góry.
				Na górze, w pierwszym rezystorze po lewej, oznaczonym wcześniej jako R cztery, jest strzałka prądu skierowana w prawo z opisem i e, obok zaznaczono też spadek napięcia u R cztery strzałką skierowaną w lewo.
				W punkcie między R cztery a R jeden znajduje się węzeł W jeden, z którego w dół biegnie rezystor R dwa z prądem i R dwa skierowanym w dół i ze strzałką napięcia u R dwa skierowaną w górę.
				W kolejnym rezystorze w górnej gałęzi, R jeden, zaznaczono prąd i R jeden skierowany w lewo oraz spadek napięcia u R jeden strzałką w prawo; węzeł po prawej stronie tego rezystora opisany jest jako W dwa.
				Z węzła W dwa w dół biegnie rezystor R trzy, przy którym strzałka prądu i R trzy skierowana jest w dół, a napięcie u R trzy zaznaczono strzałką w górę.
				Po prawej stronie dolnej gałęzi nadal znajduje się źródło prądowe z prądem j skierowanym w górę, a obok zaznaczono napięcie u j między górnym a dolnym zaciskiem źródła.
				Wewnątrz obwodu narysowano trzy pętle oczkowe: po lewej oczko O jeden obejmuje źródło napięcia, rezystor R cztery i rezystor R dwa; środkowe oczko O dwa obejmuje rezystory R jeden, R dwa i R trzy; prawe oczko O trzy obejmuje rezystor R trzy i źródło prądowe.
				Przy dolnej szynie zaznaczono opis W trzy, wskazując, że jest to wspólny węzeł odniesienia dla wszystkich gałęzi obwodu.}]{./imags/L5/fig2b.png}	
				\caption{Przykładowe zadanie.}
			\end{figure}	
		\end{column}
	\end{columns}
\end{frame} 


\begin{frame}{Równania węzłowe - przykład}
	\begin{columns}
		\begin{column}{0.4\textwidth}
			Równania węzłów:\\
			$w_1$: $i_e+i_{R1}-i_{R2}=0$\\
			$w_2$: $-i_{R1}-i_{R3}+j=0$\\
			$w_3$: $-i_e+i_{R2}+i_{R3}-j=0$
		\end{column}
		\begin{column}{0.6\textwidth}
			\begin{figure}
				\includegraphics[scale=0.08, alt={Na rysunku znajduje się ten sam obwód co poprzednio, ale uzupełniony o oznaczenia prądów, spadków napięć oraz trzy oczka oznaczone O jeden, O dwa i O trzy, a także o oznaczenia trzech węzłów W jeden, W dwa oraz dolnej szyny jako W trzy.
				Po lewej w dolnej gałęzi nadal jest źródło napięcia e, obok którego zaznaczono strzałką kierunek napięcia od dołu do góry.
				Na górze, w pierwszym rezystorze po lewej, oznaczonym wcześniej jako R cztery, jest strzałka prądu skierowana w prawo z opisem i e, obok zaznaczono też spadek napięcia u R cztery strzałką skierowaną w lewo.
				W punkcie między R cztery a R jeden znajduje się węzeł W jeden, z którego w dół biegnie rezystor R dwa z prądem i R dwa skierowanym w dół i ze strzałką napięcia u R dwa skierowaną w górę.
				W kolejnym rezystorze w górnej gałęzi, R jeden, zaznaczono prąd i R jeden skierowany w lewo oraz spadek napięcia u R jeden strzałką w prawo; węzeł po prawej stronie tego rezystora opisany jest jako W dwa.
				Z węzła W dwa w dół biegnie rezystor R trzy, przy którym strzałka prądu i R trzy skierowana jest w dół, a napięcie u R trzy zaznaczono strzałką w górę.
				Po prawej stronie dolnej gałęzi nadal znajduje się źródło prądowe z prądem j skierowanym w górę, a obok zaznaczono napięcie u j między górnym a dolnym zaciskiem źródła.
				Wewnątrz obwodu narysowano trzy pętle oczkowe: po lewej oczko O jeden obejmuje źródło napięcia, rezystor R cztery i rezystor R dwa; środkowe oczko O dwa obejmuje rezystory R jeden, R dwa i R trzy; prawe oczko O trzy obejmuje rezystor R trzy i źródło prądowe.
				Przy dolnej szynie zaznaczono opis W trzy, wskazując, że jest to wspólny węzeł odniesienia dla wszystkich gałęzi obwodu.}]{./imags/L5/fig2b.png}	
				\caption{Przykładowe zadanie.}
			\end{figure}	
		\end{column}
	\end{columns}
\end{frame} 

\begin{frame}{Równania węzłowe - przykład}
	\begin{columns}
		\begin{column}{0.4\textwidth}
			Równania węzłów:\\
			$w_1$: $i_e+i_{R1}-i_{R2}=0$\\
			$w_2$: $-i_{R1}-i_{R3}+j=0$\\
			$w_3$: $-i_e+i_{R2}+i_{R3}-j=0$\\
			Można zauważyć, że suma 3 powyższych równań jest równa zero.
			Oznacza to, że są one liniowo zależne, albo inaczej - że jedno z równań zawiera już informację zawartą w poprzednich. 
			Z tego powodu, w układzie gdzie występuje $N$ węzłów używa się równań z dowolnych $N-1$ węzłów.
		\end{column}
		\begin{column}{0.6\textwidth}
			\begin{figure}
				\includegraphics[scale=0.08, alt={Na rysunku znajduje się ten sam obwód co poprzednio, ale uzupełniony o oznaczenia prądów, spadków napięć oraz trzy oczka oznaczone O jeden, O dwa i O trzy, a także o oznaczenia trzech węzłów W jeden, W dwa oraz dolnej szyny jako W trzy.
				Po lewej w dolnej gałęzi nadal jest źródło napięcia e, obok którego zaznaczono strzałką kierunek napięcia od dołu do góry.
				Na górze, w pierwszym rezystorze po lewej, oznaczonym wcześniej jako R cztery, jest strzałka prądu skierowana w prawo z opisem i e, obok zaznaczono też spadek napięcia u R cztery strzałką skierowaną w lewo.
				W punkcie między R cztery a R jeden znajduje się węzeł W jeden, z którego w dół biegnie rezystor R dwa z prądem i R dwa skierowanym w dół i ze strzałką napięcia u R dwa skierowaną w górę.
				W kolejnym rezystorze w górnej gałęzi, R jeden, zaznaczono prąd i R jeden skierowany w lewo oraz spadek napięcia u R jeden strzałką w prawo; węzeł po prawej stronie tego rezystora opisany jest jako W dwa.
				Z węzła W dwa w dół biegnie rezystor R trzy, przy którym strzałka prądu i R trzy skierowana jest w dół, a napięcie u R trzy zaznaczono strzałką w górę.
				Po prawej stronie dolnej gałęzi nadal znajduje się źródło prądowe z prądem j skierowanym w górę, a obok zaznaczono napięcie u j między górnym a dolnym zaciskiem źródła.
				Wewnątrz obwodu narysowano trzy pętle oczkowe: po lewej oczko O jeden obejmuje źródło napięcia, rezystor R cztery i rezystor R dwa; środkowe oczko O dwa obejmuje rezystory R jeden, R dwa i R trzy; prawe oczko O trzy obejmuje rezystor R trzy i źródło prądowe.
				Przy dolnej szynie zaznaczono opis W trzy, wskazując, że jest to wspólny węzeł odniesienia dla wszystkich gałęzi obwodu.}]{./imags/L5/fig2b.png}	
				\caption{Przykładowe zadanie.}
			\end{figure}	
		\end{column}
	\end{columns}
\end{frame} 

\begin{frame}{Równania węzłowe - przykład}
	\begin{columns}
		\begin{column}{0.4\textwidth}
			Dla wygody zapiszmy równania węzłów w postaci macierzowej:
			\[
			\begin{bmatrix}
			1 & 1 & -1 & 0 \\
			0 & -1 & 0 & -1 \\
			-1 & 0 & 1 & 1
			\end{bmatrix}
			\begin{bmatrix}
			i_e \\ i_{R1} \\ i_{R2} \\ i_{R3}
			\end{bmatrix}
			=
			\begin{bmatrix}
			0 \\ -j \\ j
			\end{bmatrix}
			\]
			Bez nadmiarowego równania:
			\[
			\begin{bmatrix}
			1 & 1 & -1 & 0 \\
			0 & -1 & 0 & -1
			\end{bmatrix}
			\begin{bmatrix}
			i_e \\ i_{R1} \\ i_{R2} \\ i_{R3}
			\end{bmatrix}
			=
			\begin{bmatrix}
			0 \\ -j
			\end{bmatrix}
			\]
		\end{column}
		\begin{column}{0.6\textwidth}
			\begin{figure}
				\includegraphics[scale=0.08, alt={Na rysunku znajduje się ten sam obwód co poprzednio, ale uzupełniony o oznaczenia prądów, spadków napięć oraz trzy oczka oznaczone O jeden, O dwa i O trzy, a także o oznaczenia trzech węzłów W jeden, W dwa oraz dolnej szyny jako W trzy.
				Po lewej w dolnej gałęzi nadal jest źródło napięcia e, obok którego zaznaczono strzałką kierunek napięcia od dołu do góry.
				Na górze, w pierwszym rezystorze po lewej, oznaczonym wcześniej jako R cztery, jest strzałka prądu skierowana w prawo z opisem i e, obok zaznaczono też spadek napięcia u R cztery strzałką skierowaną w lewo.
				W punkcie między R cztery a R jeden znajduje się węzeł W jeden, z którego w dół biegnie rezystor R dwa z prądem i R dwa skierowanym w dół i ze strzałką napięcia u R dwa skierowaną w górę.
				W kolejnym rezystorze w górnej gałęzi, R jeden, zaznaczono prąd i R jeden skierowany w lewo oraz spadek napięcia u R jeden strzałką w prawo; węzeł po prawej stronie tego rezystora opisany jest jako W dwa.
				Z węzła W dwa w dół biegnie rezystor R trzy, przy którym strzałka prądu i R trzy skierowana jest w dół, a napięcie u R trzy zaznaczono strzałką w górę.
				Po prawej stronie dolnej gałęzi nadal znajduje się źródło prądowe z prądem j skierowanym w górę, a obok zaznaczono napięcie u j między górnym a dolnym zaciskiem źródła.
				Wewnątrz obwodu narysowano trzy pętle oczkowe: po lewej oczko O jeden obejmuje źródło napięcia, rezystor R cztery i rezystor R dwa; środkowe oczko O dwa obejmuje rezystory R jeden, R dwa i R trzy; prawe oczko O trzy obejmuje rezystor R trzy i źródło prądowe.
				Przy dolnej szynie zaznaczono opis W trzy, wskazując, że jest to wspólny węzeł odniesienia dla wszystkich gałęzi obwodu.}]{./imags/L5/fig2b.png}	
				\caption{Przykładowe zadanie.}
			\end{figure}	
		\end{column}
	\end{columns}
\end{frame} 

\begin{frame}{Równania węzłowe - przykład}
	\begin{columns}
		\begin{column}{0.4\textwidth}
			Zajmijmy się równaniami oczek.
			W układzie występują 3 proste oczka, ich równania układa się korzystając z NPK.
			Na ogół przyjmuje się że oczka "obracają się" zgodnie z kierunkiem wskazówek zegara, jednak przyjęcie odwrotnego kierunku zmieni jedynie znaki, a rozwiązanie wyjdzie identyczne.
		\end{column}
		\begin{column}{0.6\textwidth}
			\begin{figure}
				\includegraphics[scale=0.08, alt={Na rysunku znajduje się ten sam obwód co poprzednio, ale uzupełniony o oznaczenia prądów, spadków napięć oraz trzy oczka oznaczone O jeden, O dwa i O trzy, a także o oznaczenia trzech węzłów W jeden, W dwa oraz dolnej szyny jako W trzy.
				Po lewej w dolnej gałęzi nadal jest źródło napięcia e, obok którego zaznaczono strzałką kierunek napięcia od dołu do góry.
				Na górze, w pierwszym rezystorze po lewej, oznaczonym wcześniej jako R cztery, jest strzałka prądu skierowana w prawo z opisem i e, obok zaznaczono też spadek napięcia u R cztery strzałką skierowaną w lewo.
				W punkcie między R cztery a R jeden znajduje się węzeł W jeden, z którego w dół biegnie rezystor R dwa z prądem i R dwa skierowanym w dół i ze strzałką napięcia u R dwa skierowaną w górę.
				W kolejnym rezystorze w górnej gałęzi, R jeden, zaznaczono prąd i R jeden skierowany w lewo oraz spadek napięcia u R jeden strzałką w prawo; węzeł po prawej stronie tego rezystora opisany jest jako W dwa.
				Z węzła W dwa w dół biegnie rezystor R trzy, przy którym strzałka prądu i R trzy skierowana jest w dół, a napięcie u R trzy zaznaczono strzałką w górę.
				Po prawej stronie dolnej gałęzi nadal znajduje się źródło prądowe z prądem j skierowanym w górę, a obok zaznaczono napięcie u j między górnym a dolnym zaciskiem źródła.
				Wewnątrz obwodu narysowano trzy pętle oczkowe: po lewej oczko O jeden obejmuje źródło napięcia, rezystor R cztery i rezystor R dwa; środkowe oczko O dwa obejmuje rezystory R jeden, R dwa i R trzy; prawe oczko O trzy obejmuje rezystor R trzy i źródło prądowe.
				Przy dolnej szynie zaznaczono opis W trzy, wskazując, że jest to wspólny węzeł odniesienia dla wszystkich gałęzi obwodu.}]{./imags/L5/fig2b.png}	
				\caption{Przykładowe zadanie.}
			\end{figure}	
		\end{column}
	\end{columns}
\end{frame} 

\begin{frame}{Równania węzłowe - przykład}
	\begin{columns}
		\begin{column}{0.4\textwidth}
			Równania oczek:\\
			$O_1$: $e-U_{R4}-U_{R2}=0$\\
			$O_2$: $U_{R2}+U_{R1}-U_{R3}=0$\\
			$O_3$: $U_{R3}-u_j=0$\\
			W tym przypadku wszystkie równania są istotne.
			Można również budować oczka przekraczające elementy, np. połączone oczko $O_1$ i $O_2$, jednak w ogólnym przypadku nie jest to konieczne.
		\end{column}
		\begin{column}{0.6\textwidth}
			\begin{figure}
				\includegraphics[scale=0.08, alt={Na rysunku znajduje się ten sam obwód co poprzednio, ale uzupełniony o oznaczenia prądów, spadków napięć oraz trzy oczka oznaczone O jeden, O dwa i O trzy, a także o oznaczenia trzech węzłów W jeden, W dwa oraz dolnej szyny jako W trzy.
				Po lewej w dolnej gałęzi nadal jest źródło napięcia e, obok którego zaznaczono strzałką kierunek napięcia od dołu do góry.
				Na górze, w pierwszym rezystorze po lewej, oznaczonym wcześniej jako R cztery, jest strzałka prądu skierowana w prawo z opisem i e, obok zaznaczono też spadek napięcia u R cztery strzałką skierowaną w lewo.
				W punkcie między R cztery a R jeden znajduje się węzeł W jeden, z którego w dół biegnie rezystor R dwa z prądem i R dwa skierowanym w dół i ze strzałką napięcia u R dwa skierowaną w górę.
				W kolejnym rezystorze w górnej gałęzi, R jeden, zaznaczono prąd i R jeden skierowany w lewo oraz spadek napięcia u R jeden strzałką w prawo; węzeł po prawej stronie tego rezystora opisany jest jako W dwa.
				Z węzła W dwa w dół biegnie rezystor R trzy, przy którym strzałka prądu i R trzy skierowana jest w dół, a napięcie u R trzy zaznaczono strzałką w górę.
				Po prawej stronie dolnej gałęzi nadal znajduje się źródło prądowe z prądem j skierowanym w górę, a obok zaznaczono napięcie u j między górnym a dolnym zaciskiem źródła.
				Wewnątrz obwodu narysowano trzy pętle oczkowe: po lewej oczko O jeden obejmuje źródło napięcia, rezystor R cztery i rezystor R dwa; środkowe oczko O dwa obejmuje rezystory R jeden, R dwa i R trzy; prawe oczko O trzy obejmuje rezystor R trzy i źródło prądowe.
				Przy dolnej szynie zaznaczono opis W trzy, wskazując, że jest to wspólny węzeł odniesienia dla wszystkich gałęzi obwodu.}]{./imags/L5/fig2b.png}	
				\caption{Przykładowe zadanie.}
			\end{figure}	
		\end{column}
	\end{columns}
\end{frame} 

\begin{frame}{Równania węzłowe - przykład}
	\begin{columns}
		\begin{column}{0.4\textwidth}
			Można zauważyć, że równania oczek oparte są o napięcia, które w przypadku rezystorów można uzależnić od prądów:\\
			$U_{R1}=i_{R1} R_1$\\
			$U_{R2}=i_{R2} R_2$\\
			$U_{R3}=i_{R3} R_3$\\
			$U_{R4}=i_{e} R_4$\\
			Co z z kolei pozwala na modyfikację równań oczek:\\
			$O_1$: $e-i_{e} R_4-i_{R2} R_2=0$\\
			$O_2$: $i_{R2} R_2+i_{R1} R_1-i_{R3} R_3=0$\\
			$O_3$: $i_{R3} R_3-u_j=0$\\
		\end{column}
		\begin{column}{0.6\textwidth}
			\begin{figure}
				\includegraphics[scale=0.08, alt={Na rysunku znajduje się ten sam obwód co poprzednio, ale uzupełniony o oznaczenia prądów, spadków napięć oraz trzy oczka oznaczone O jeden, O dwa i O trzy, a także o oznaczenia trzech węzłów W jeden, W dwa oraz dolnej szyny jako W trzy.
				Po lewej w dolnej gałęzi nadal jest źródło napięcia e, obok którego zaznaczono strzałką kierunek napięcia od dołu do góry.
				Na górze, w pierwszym rezystorze po lewej, oznaczonym wcześniej jako R cztery, jest strzałka prądu skierowana w prawo z opisem i e, obok zaznaczono też spadek napięcia u R cztery strzałką skierowaną w lewo.
				W punkcie między R cztery a R jeden znajduje się węzeł W jeden, z którego w dół biegnie rezystor R dwa z prądem i R dwa skierowanym w dół i ze strzałką napięcia u R dwa skierowaną w górę.
				W kolejnym rezystorze w górnej gałęzi, R jeden, zaznaczono prąd i R jeden skierowany w lewo oraz spadek napięcia u R jeden strzałką w prawo; węzeł po prawej stronie tego rezystora opisany jest jako W dwa.
				Z węzła W dwa w dół biegnie rezystor R trzy, przy którym strzałka prądu i R trzy skierowana jest w dół, a napięcie u R trzy zaznaczono strzałką w górę.
				Po prawej stronie dolnej gałęzi nadal znajduje się źródło prądowe z prądem j skierowanym w górę, a obok zaznaczono napięcie u j między górnym a dolnym zaciskiem źródła.
				Wewnątrz obwodu narysowano trzy pętle oczkowe: po lewej oczko O jeden obejmuje źródło napięcia, rezystor R cztery i rezystor R dwa; środkowe oczko O dwa obejmuje rezystory R jeden, R dwa i R trzy; prawe oczko O trzy obejmuje rezystor R trzy i źródło prądowe.
				Przy dolnej szynie zaznaczono opis W trzy, wskazując, że jest to wspólny węzeł odniesienia dla wszystkich gałęzi obwodu.}]{./imags/L5/fig2b.png}	
				\caption{Przykładowe zadanie.}
			\end{figure}	
		\end{column}
	\end{columns}
\end{frame} 


\begin{frame}{Równania węzłowe - przykład}
	\begin{columns}
		\begin{column}{0.4\textwidth}
			Ponownie, równania można przedstawić w postaci macierzowej:
			\[
			\begin{bmatrix}
			-R_4 & 0 & -R_2 & 0 & 0 \\
			0 & R_1 & R_2 & -R_3 & 0 \\
			0 & 0 & 0 & R_3 & -1
			\end{bmatrix}
			\begin{bmatrix}
			i_e \\ i_{R1} \\ i_{R2} \\ i_{R3} \\ u_j
			\end{bmatrix}
			=
			\begin{bmatrix}
			-e \\ 0 \\ 0
			\end{bmatrix}
			\]
		\end{column}
		\begin{column}{0.6\textwidth}
			\begin{figure}
				\includegraphics[scale=0.08, alt={Na rysunku znajduje się ten sam obwód co poprzednio, ale uzupełniony o oznaczenia prądów, spadków napięć oraz trzy oczka oznaczone O jeden, O dwa i O trzy, a także o oznaczenia trzech węzłów W jeden, W dwa oraz dolnej szyny jako W trzy.
				Po lewej w dolnej gałęzi nadal jest źródło napięcia e, obok którego zaznaczono strzałką kierunek napięcia od dołu do góry.
				Na górze, w pierwszym rezystorze po lewej, oznaczonym wcześniej jako R cztery, jest strzałka prądu skierowana w prawo z opisem i e, obok zaznaczono też spadek napięcia u R cztery strzałką skierowaną w lewo.
				W punkcie między R cztery a R jeden znajduje się węzeł W jeden, z którego w dół biegnie rezystor R dwa z prądem i R dwa skierowanym w dół i ze strzałką napięcia u R dwa skierowaną w górę.
				W kolejnym rezystorze w górnej gałęzi, R jeden, zaznaczono prąd i R jeden skierowany w lewo oraz spadek napięcia u R jeden strzałką w prawo; węzeł po prawej stronie tego rezystora opisany jest jako W dwa.
				Z węzła W dwa w dół biegnie rezystor R trzy, przy którym strzałka prądu i R trzy skierowana jest w dół, a napięcie u R trzy zaznaczono strzałką w górę.
				Po prawej stronie dolnej gałęzi nadal znajduje się źródło prądowe z prądem j skierowanym w górę, a obok zaznaczono napięcie u j między górnym a dolnym zaciskiem źródła.
				Wewnątrz obwodu narysowano trzy pętle oczkowe: po lewej oczko O jeden obejmuje źródło napięcia, rezystor R cztery i rezystor R dwa; środkowe oczko O dwa obejmuje rezystory R jeden, R dwa i R trzy; prawe oczko O trzy obejmuje rezystor R trzy i źródło prądowe.
				Przy dolnej szynie zaznaczono opis W trzy, wskazując, że jest to wspólny węzeł odniesienia dla wszystkich gałęzi obwodu.}]{./imags/L5/fig2b.png}	
				\caption{Przykładowe zadanie.}
			\end{figure}	
		\end{column}
	\end{columns}
\end{frame} 


\begin{frame}{Równania węzłowe - przykład}
	\begin{columns}
		\begin{column}{0.4\textwidth}
			Łącząc równania uzyskane z węzłów i oczek uzyskuje się równanie macierzowe postaci:
			\[
			\begin{bmatrix}
			- R_4 & 0   & - R_2 & 0     & 0 \\
			0     & R_1 &   R_2 & - R_3 & 0 \\
			0     & 0   &   0   & R_3   & -1 \\
			1     & 1   &  -1   & 0     & 0 \\
			0     & -1  &   0   & -1    & 0
			\end{bmatrix}
			\begin{bmatrix}
			i_e \\ i_{R1} \\ i_{R2} \\ i_{R3} \\ u_j
			\end{bmatrix}
			=
			\begin{bmatrix}
			- e \\ 0 \\ 0 \\ 0 \\ - j
			\end{bmatrix}
			\]
		\end{column}
		\begin{column}{0.6\textwidth}
			\begin{figure}
				\includegraphics[scale=0.08, alt={Na rysunku znajduje się ten sam obwód co poprzednio, ale uzupełniony o oznaczenia prądów, spadków napięć oraz trzy oczka oznaczone O jeden, O dwa i O trzy, a także o oznaczenia trzech węzłów W jeden, W dwa oraz dolnej szyny jako W trzy.
				Po lewej w dolnej gałęzi nadal jest źródło napięcia e, obok którego zaznaczono strzałką kierunek napięcia od dołu do góry.
				Na górze, w pierwszym rezystorze po lewej, oznaczonym wcześniej jako R cztery, jest strzałka prądu skierowana w prawo z opisem i e, obok zaznaczono też spadek napięcia u R cztery strzałką skierowaną w lewo.
				W punkcie między R cztery a R jeden znajduje się węzeł W jeden, z którego w dół biegnie rezystor R dwa z prądem i R dwa skierowanym w dół i ze strzałką napięcia u R dwa skierowaną w górę.
				W kolejnym rezystorze w górnej gałęzi, R jeden, zaznaczono prąd i R jeden skierowany w lewo oraz spadek napięcia u R jeden strzałką w prawo; węzeł po prawej stronie tego rezystora opisany jest jako W dwa.
				Z węzła W dwa w dół biegnie rezystor R trzy, przy którym strzałka prądu i R trzy skierowana jest w dół, a napięcie u R trzy zaznaczono strzałką w górę.
				Po prawej stronie dolnej gałęzi nadal znajduje się źródło prądowe z prądem j skierowanym w górę, a obok zaznaczono napięcie u j między górnym a dolnym zaciskiem źródła.
				Wewnątrz obwodu narysowano trzy pętle oczkowe: po lewej oczko O jeden obejmuje źródło napięcia, rezystor R cztery i rezystor R dwa; środkowe oczko O dwa obejmuje rezystory R jeden, R dwa i R trzy; prawe oczko O trzy obejmuje rezystor R trzy i źródło prądowe.
				Przy dolnej szynie zaznaczono opis W trzy, wskazując, że jest to wspólny węzeł odniesienia dla wszystkich gałęzi obwodu.}]{./imags/L5/fig2b.png}	
				\caption{Przykładowe zadanie.}
			\end{figure}	
		\end{column}
	\end{columns}
\end{frame} 


\begin{frame}{Równania węzłowe - przykład}
	\begin{columns}
		\begin{column}{0.4\textwidth}
			Równanie to posiada jednoznaczne rozwiązanie, które wyznaczyć można poprzez wykonanie operacji:\\
			\[
			\begin{bmatrix}
			i_e \\ i_{R1} \\ i_{R2} \\ i_{R3} \\ u_j
			\end{bmatrix}
			=
			\begin{bmatrix}
			- R_4 & 0   & - R_2 & 0     & 0 \\
			0     & R_1 &   R_2 & - R_3 & 0 \\
			0     & 0   &   0   & R_3   & -1 \\
			1     & 1   &  -1   & 0     & 0 \\
			0     & -1  &   0   & -1    & 0
			\end{bmatrix}^{-1}
			\begin{bmatrix}
			- e \\ 0 \\ 0 \\ 0 \\ - j
			\end{bmatrix}
			\]
		\end{column}
		\begin{column}{0.6\textwidth}
			\begin{figure}
				\includegraphics[scale=0.08, alt={Na rysunku znajduje się ten sam obwód co poprzednio, ale uzupełniony o oznaczenia prądów, spadków napięć oraz trzy oczka oznaczone O jeden, O dwa i O trzy, a także o oznaczenia trzech węzłów W jeden, W dwa oraz dolnej szyny jako W trzy.
				Po lewej w dolnej gałęzi nadal jest źródło napięcia e, obok którego zaznaczono strzałką kierunek napięcia od dołu do góry.
				Na górze, w pierwszym rezystorze po lewej, oznaczonym wcześniej jako R cztery, jest strzałka prądu skierowana w prawo z opisem i e, obok zaznaczono też spadek napięcia u R cztery strzałką skierowaną w lewo.
				W punkcie między R cztery a R jeden znajduje się węzeł W jeden, z którego w dół biegnie rezystor R dwa z prądem i R dwa skierowanym w dół i ze strzałką napięcia u R dwa skierowaną w górę.
				W kolejnym rezystorze w górnej gałęzi, R jeden, zaznaczono prąd i R jeden skierowany w lewo oraz spadek napięcia u R jeden strzałką w prawo; węzeł po prawej stronie tego rezystora opisany jest jako W dwa.
				Z węzła W dwa w dół biegnie rezystor R trzy, przy którym strzałka prądu i R trzy skierowana jest w dół, a napięcie u R trzy zaznaczono strzałką w górę.
				Po prawej stronie dolnej gałęzi nadal znajduje się źródło prądowe z prądem j skierowanym w górę, a obok zaznaczono napięcie u j między górnym a dolnym zaciskiem źródła.
				Wewnątrz obwodu narysowano trzy pętle oczkowe: po lewej oczko O jeden obejmuje źródło napięcia, rezystor R cztery i rezystor R dwa; środkowe oczko O dwa obejmuje rezystory R jeden, R dwa i R trzy; prawe oczko O trzy obejmuje rezystor R trzy i źródło prądowe.
				Przy dolnej szynie zaznaczono opis W trzy, wskazując, że jest to wspólny węzeł odniesienia dla wszystkich gałęzi obwodu.}]{./imags/L5/fig2b.png}	
				\caption{Przykładowe zadanie.}
			\end{figure}	
		\end{column}
	\end{columns}
\end{frame} 

\begin{frame}[fragile]{Równania węzłowe - przykład}
Rozwiązanie tego typu równania można uzyskać dowolnym, ulubionym programem
obliczającym równania macierzowe:\\
Macierz A w Matlabie:
	\begin{lstlisting}[style=python-wcag]
		>> A=[-4 0 -2 0 0; 0 1 2 -3 0; 0 0 0 3 -1; 1 1 -1 0 0; 0 -1 0 -1 0]
		A =
		
		  -4   0  -2   0   0
		   0   1   2  -3   0
		   0   0   0   3  -1
		   1   1  -1   0   0
		   0  -1   0  -1   0
	\end{lstlisting}
\end{frame}

\begin{frame}[fragile]{Równania węzłowe - przykład}
Wektor y w Matlabie:
	\begin{lstlisting}[style=python-wcag]
		>> y=[-5; 0; 0; 0; -1]
		y =
		
		  -5
		   0
		   0
		   0
		  -1
	\end{lstlisting}
\end{frame}

\begin{frame}[fragile]{Równania węzłowe - przykład}
Rozwiązanie równania:
	\begin{lstlisting}[style=python-wcag]
		>> A^-1*y
		ans =
		
		   0.7500
		   0.2500
		   1.0000
		   0.7500
		   2.2500
	\end{lstlisting}
\end{frame}

\begin{frame}{Przykład - symulacja}
	\begin{block}{Przykład - symulacja}
		\url{https://www.falstad.com/circuit/circuitjs.html?ctz=CQAgjCAMB0l3BWcMBMcUHYMGZIA4UA2ATmIxG2xCQBYqEBTAWjDACgBLEGwmkFPHm69whSFHBsAThWyFR4tELBiJ7GUoUg84lePE1p-fFs0Chio5Xl7Z88xOxsA7na08+tyC+F8H1-kEoHwCHMyDvV3ChHUCLNgA3bUUg2K8JGnEkfShoBDYgA}
	\end{block}
\end{frame}
