\sect{Wykład 1}

\begin{frame}{Wielkości fizyczne}
	\begin{itemize}
		\item[\bt] Ładunek
		\item[\bt] Natężenie
		\item[\bt] Napięcie
	\end{itemize}
\end{frame}

\begin{frame}{Prawo Ohma}
	\begin{itemize}
		\item[\bt] Rezystancja Om/Siemens
	\end{itemize}
\end{frame}

\begin{frame}{Energia}
	\begin{itemize}
		\item[\bt] Dżul/Wat
	\end{itemize}
\end{frame}

\begin{frame}{Definicja obwodu elektrycznego}
	Placeholder
\end{frame}

\begin{frame}{Przedrostki + notacja wykładnicza}
	Placeholder
\end{frame}

\begin{frame}{Prądowe prawo Kirchoffa (PPK)}
	Placeholder
\end{frame}

\begin{frame}{Napięciowe prawi Kirchofa (NPK)}
	Placeholder
\end{frame}

\begin{frame}{Zasada zachowania energii (zasada Tellgena)}
	Placeholder
\end{frame}

\begin{frame}{Elementy w obwodzie}
	\begin{itemize}
		\item[\bt] Rezystory
		\item[\bt] Źródła niezależne
		\item[\bt] Żródła zależne
		\item[\bt] Wzmacniacze operacyjne
    \end{itemize}
\end{frame}

\begin{frame}{Definicje SLS}
	Placeholder
\end{frame}

\begin{frame}{Strzałkowanie w obwodzie}
	Placeholder
\end{frame}

\begin{frame}{Omówienie źródła napięciowego}
	Placeholder
\end{frame} 

\begin{frame}{Omówienie źródła prądowego}
	Placeholder
\end{frame} 

\sect{Wykład 2}

\begin{frame}{Łączenie rezystorów}
	Placeholder
\end{frame} 

\begin{frame}{Źródło Thevenina}
	Placeholder
\end{frame} 

\begin{frame}{Źródło Nortona}
	Placeholder
\end{frame} 

\begin{frame}{Połączenie G-T}
	Placeholder
\end{frame} 

\sect{Wykład 3}
\begin{frame}{Bilans mocy}
	Placeholder
\end{frame} 

\begin{frame}{Optymalny pobór mocy}
	Placeholder
\end{frame} 

\sect{Wykład 4}

\begin{frame}{Zasada superpozycji}
	Placeholder
\end{frame} 

\begin{frame}{Metoda zamiany źródeł}
	Placeholder
\end{frame} 

\sect{Wykład 5}
\begin{frame}{Równania oczkowe}
	Placeholder
\end{frame} 

\begin{frame}{Rówania węzłowe}
	Placeholder
\end{frame} 

\begin{frame}{Ogólna metoda rozwiązywania schematów}
	Placeholder
\end{frame} 

\sect{Wykład 6}
\begin{frame}{Źródła sterowane (4 sztuki)}
	Placeholder
\end{frame} 

\begin{frame}{Wzmacniacz operacyjny}
	Placeholder
\end{frame} 

\sect{Wykład 7}
\begin{frame}{Metoda prądów oczkowych}
	Placeholder
\end{frame} 

\begin{frame}{Metoda potencjałów węzłowych}
	Placeholder
\end{frame} 
 
 
 