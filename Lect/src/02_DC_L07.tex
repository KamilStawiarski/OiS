\sect{Wykład 7}

\begin{frame}{Wzmacniacz operacyjny}
	\begin{columns}
		\begin{column}{0.6\textwidth}
			Wzmacniacz operacyjny to scalony wzmacniacz różnicowy o bardzo dużym wzmocnieniu napięciowym, który prawie zawsze pracuje ze sprzężeniem zwrotnym (ujemnym lub dodatnim). Jest podstawowym blokiem elektroniki analogowej do budowania szerokiej gamy układów elektronicznych i idealnym narzędziem do operacji matematycznych na sygnałach analogowych.
		\end{column}
		\begin{column}{0.4\textwidth}
			\begin{figure}
				\includegraphics[scale=0.08, alt={Na rysunku przedstawiono dwa symbole wzmacniaczy operacyjnych, jeden nad drugim.
				Górny symbol to trójkąt zwrócony wierzchołkiem w prawo; jego lewy bok jest pionowy. Z lewej strony, w połowie wysokości tego boku, wychodzą dwa krótkie, poziome odcinki przewodów: górny do punktu na obudowie trójkąta oznaczonego znakiem plus, dolny do punktu oznaczonego znakiem minus. Po prawej stronie trójkąta, w jego wierzchołku, wychodzi pojedynczy poziomy przewód - wyjście wzmacniacza.
				Dolny symbol jest identyczny geometrycznie, ale zamieniono miejscami oznaczenia wejść: w połowie lewego boku trójkąta również są dwa poziome wejścia, przy czym górne wejście oznaczone jest minusem, a dolne plusem. Z wierzchołka po prawej stronie wychodzi jeden poziomy przewód stanowiący wyjście drugiego wzmacniacza.}]{./imags/L7/OpAmp.png}	
				\caption{Dwa wzmacniacze operacyjne.}
			\end{figure}	
		\end{column}
	\end{columns}
\end{frame} 

\begin{frame}{Wzmacniacz operacyjny}
	\begin{columns}
		\begin{column}{0.6\textwidth}
			\textbf{Budowa wewnętrzna}\\
			Wzmacniacz operacyjny posiada wejście diferencyj­ne z dwoma terminalami: nieodwracającym (+) i odwracającym (-), wzmacniacz różnicowy o ogromnym wzmocnieniu otwartopętlowym (A0), stadiu wyjściowy o niskiej impedancji oraz zasilanie dwupolarne (dodatnie i ujemne napięcie zasilania).
		\end{column}
		\begin{column}{0.4\textwidth}
			\begin{figure}
				\includegraphics[scale=0.08, alt={Na rysunku przedstawiono dwa symbole wzmacniaczy operacyjnych, jeden nad drugim.
				Górny symbol to trójkąt zwrócony wierzchołkiem w prawo; jego lewy bok jest pionowy. Z lewej strony, w połowie wysokości tego boku, wychodzą dwa krótkie, poziome odcinki przewodów: górny do punktu na obudowie trójkąta oznaczonego znakiem plus, dolny do punktu oznaczonego znakiem minus. Po prawej stronie trójkąta, w jego wierzchołku, wychodzi pojedynczy poziomy przewód - wyjście wzmacniacza.
				Dolny symbol jest identyczny geometrycznie, ale zamieniono miejscami oznaczenia wejść: w połowie lewego boku trójkąta również są dwa poziome wejścia, przy czym górne wejście oznaczone jest minusem, a dolne plusem. Z wierzchołka po prawej stronie wychodzi jeden poziomy przewód stanowiący wyjście drugiego wzmacniacza.}]{./imags/L7/OpAmp.png}	
				\caption{Dwa wzmacniacze operacyjne.}
			\end{figure}	
		\end{column}
	\end{columns}
\end{frame} 

\begin{frame}{Wzmacniacz operacyjny}
	\begin{columns}
		\begin{column}{0.6\textwidth}
			\textbf{Symbol i zaciski}\\
			Na schemacie wzmacniacz operacyjny jest reprezentowany jako trójkąt zwrócony wierzchołkiem w prawo. Ma dwa zaciski wejściowe z lewej: plus (nieodwracający) i minus (odwracający), jeden zaciski wyjściowy po prawej stronie (wierzchołek), zaciski zasilania: plus V (dodatnie) i minus V (ujemne), a czasami dodatkowe zaciski do kalibracji offsetu i stabilizacji.
		\end{column}
		\begin{column}{0.4\textwidth}
			\begin{figure}
				\includegraphics[scale=0.08, alt={Na rysunku przedstawiono dwa symbole wzmacniaczy operacyjnych, jeden nad drugim.
				Górny symbol to trójkąt zwrócony wierzchołkiem w prawo; jego lewy bok jest pionowy. Z lewej strony, w połowie wysokości tego boku, wychodzą dwa krótkie, poziome odcinki przewodów: górny do punktu na obudowie trójkąta oznaczonego znakiem plus, dolny do punktu oznaczonego znakiem minus. Po prawej stronie trójkąta, w jego wierzchołku, wychodzi pojedynczy poziomy przewód - wyjście wzmacniacza.
				Dolny symbol jest identyczny geometrycznie, ale zamieniono miejscami oznaczenia wejść: w połowie lewego boku trójkąta również są dwa poziome wejścia, przy czym górne wejście oznaczone jest minusem, a dolne plusem. Z wierzchołka po prawej stronie wychodzi jeden poziomy przewód stanowiący wyjście drugiego wzmacniacza.}]{./imags/L7/OpAmp.png}	
				\caption{Dwa wzmacniacze operacyjne.}
			\end{figure}	
		\end{column}
	\end{columns}
\end{frame} 


\begin{frame}{Wzmacniacz operacyjny}
	\begin{columns}
		\begin{column}{0.6\textwidth}
			\textbf{Sprzężenie zwrotne – kluczowa koncepcja}\\
			Bez sprzężenia zwrotnego wzmocnienie otwartopętlowe (A0) jest niepraktyczne i bardzo niestabilne.
			Ze sprzężeniem ujemnym kontrolowane wzmocnienie zamkniętopętlowe zależy od komponentów zewnętrznych. Ze sprzężeniem dodatnim otrzymujemy oscylatory, komparatory i generatory funkcji.
			W praktyce prawie zawsze pracuje w trybie sprzężenia zwrotnego, najczęściej ujemnego.
		\end{column}
		\begin{column}{0.4\textwidth}
			\begin{figure}
				\includegraphics[scale=0.08, alt={Na rysunku przedstawiono dwa symbole wzmacniaczy operacyjnych, jeden nad drugim.
				Górny symbol to trójkąt zwrócony wierzchołkiem w prawo; jego lewy bok jest pionowy. Z lewej strony, w połowie wysokości tego boku, wychodzą dwa krótkie, poziome odcinki przewodów: górny do punktu na obudowie trójkąta oznaczonego znakiem plus, dolny do punktu oznaczonego znakiem minus. Po prawej stronie trójkąta, w jego wierzchołku, wychodzi pojedynczy poziomy przewód - wyjście wzmacniacza.
				Dolny symbol jest identyczny geometrycznie, ale zamieniono miejscami oznaczenia wejść: w połowie lewego boku trójkąta również są dwa poziome wejścia, przy czym górne wejście oznaczone jest minusem, a dolne plusem. Z wierzchołka po prawej stronie wychodzi jeden poziomy przewód stanowiący wyjście drugiego wzmacniacza.}]{./imags/L7/OpAmp.png}	
				\caption{Dwa wzmacniacze operacyjne.}
			\end{figure}	
		\end{column}
	\end{columns}
\end{frame} 

\begin{frame}{Wzmacniacz operacyjny}
	\begin{columns}
		\begin{column}{0.6\textwidth}
			\textbf{Zastosowania praktyczne}\\
			Wzmacniacz operacyjny jest stosowany w przetwarzaniu sygnałów: wzmacnianiu, filtrowaniu i porównywaniu.
			Wykorzystuje się go do operacji matematycznych: dodawania, odejmowania, całkowania i różniczkowania.
			Znajduje zastosowanie w zasilaczach do stabilizacji napięcia i prądu, w audio do wzmacniania mikrofonów, kontroli głośności i equalizerów, a także w miernictwie do pomiarów prądu fotodiod, przetwornic pomiarowych i innych zastosowań.
		\end{column}
		\begin{column}{0.4\textwidth}
			\begin{figure}
				\includegraphics[scale=0.08, alt={Na rysunku przedstawiono dwa symbole wzmacniaczy operacyjnych, jeden nad drugim.
				Górny symbol to trójkąt zwrócony wierzchołkiem w prawo; jego lewy bok jest pionowy. Z lewej strony, w połowie wysokości tego boku, wychodzą dwa krótkie, poziome odcinki przewodów: górny do punktu na obudowie trójkąta oznaczonego znakiem plus, dolny do punktu oznaczonego znakiem minus. Po prawej stronie trójkąta, w jego wierzchołku, wychodzi pojedynczy poziomy przewód - wyjście wzmacniacza.
				Dolny symbol jest identyczny geometrycznie, ale zamieniono miejscami oznaczenia wejść: w połowie lewego boku trójkąta również są dwa poziome wejścia, przy czym górne wejście oznaczone jest minusem, a dolne plusem. Z wierzchołka po prawej stronie wychodzi jeden poziomy przewód stanowiący wyjście drugiego wzmacniacza.}]{./imags/L7/OpAmp.png}	
				\caption{Dwa wzmacniacze operacyjne.}
			\end{figure}	
		\end{column}
	\end{columns}
\end{frame} 

\begin{frame}{Wzmacniacz operacyjny}
	\begin{columns}
		\begin{column}{0.6\textwidth}
			\textbf{Idealny wzmacniacz operacyjny ma zestaw założonych cech, które bardzo upraszczają analizę obwodów:}\\
			Nieskończenie duże wzmocnienie otwartopętlowe: różnica napięć między wejściami jest w praktyce równa zero w pracy liniowej, bo nawet bardzo mała różnica zostałaby "nieograniczenie" wzmocniona.\\
			Nieskończenie duża rezystancja wejściowa: do wejść nie wpływa żaden prąd, więc układ poprzedzający nie jest obciążany.\\
			Zerowa rezystancja wyjściowa: wyjście może wymuszać dowolny prąd, utrzymując zadane napięcie niezależnie od obciążenia.\\
			Nieskończona szybkość narastania (slew rate): wyjście może zmieniać napięcie natychmiastowo, bez opóźnień.\\
			Brak napięcia offsetu wejściowego: przy jednakowych napięciach na obu wejściach wyjście jest idealnie równe zero.
		\end{column}
		\begin{column}{0.4\textwidth}
			\begin{figure}
				\includegraphics[scale=0.08, alt={Na rysunku przedstawiono dwa symbole wzmacniaczy operacyjnych, jeden nad drugim.
				Górny symbol to trójkąt zwrócony wierzchołkiem w prawo; jego lewy bok jest pionowy. Z lewej strony, w połowie wysokości tego boku, wychodzą dwa krótkie, poziome odcinki przewodów: górny do punktu na obudowie trójkąta oznaczonego znakiem plus, dolny do punktu oznaczonego znakiem minus. Po prawej stronie trójkąta, w jego wierzchołku, wychodzi pojedynczy poziomy przewód - wyjście wzmacniacza.
				Dolny symbol jest identyczny geometrycznie, ale zamieniono miejscami oznaczenia wejść: w połowie lewego boku trójkąta również są dwa poziome wejścia, przy czym górne wejście oznaczone jest minusem, a dolne plusem. Z wierzchołka po prawej stronie wychodzi jeden poziomy przewód stanowiący wyjście drugiego wzmacniacza.}]{./imags/L7/OpAmp.png}	
				\caption{Dwa wzmacniacze operacyjne.}
			\end{figure}	
		\end{column}
	\end{columns}
\end{frame} 

\begin{frame}{Wzmacniacz operacyjny}
	\textbf{Wzmacniacz operacyjny jako komparator}\\
	\url{https://www.falstad.com/circuit/circuitjs.html?ctz=CQAgjCAMB0l3BWcMBsBmALAJhQdgWgJwAcYKCFICkVVNCApgLRhgBQAhiFgitxjTTEM-GsXBIWSMPHgg0IETNlswuLN17csIrJHFYdUELnm0aF6JJAA1APYAbAC4cA5g1XrNfHj-3fjUxEkCygrEABSLAAhW0cXdzYAeXlhUUU0DR0aCEg2IA}
\end{frame} 

\begin{frame}{Wzmacniacz operacyjny}
	\textbf{Proste sprzężenie zwrotne}\\
	\url{https://www.falstad.com/circuit/circuitjs.html?ctz=CQAgjCAMB0l3BWcZYA4BsB2ATNzZsEBOVA9EBSCiqhAUwFowwAoAQ3CO3E3IGZsVMLxCpwSJkjDx4IPtGxElWeAKVF0-aH3AzILYdzBdORgCxnTUEJjnbqVR9CQBSbACEQANQD2AGwAXNgBzOhYAeTlBHn50MWFyCH0AdyihEQEhAigWVOMjbPzwdEdcq2ZuTORufSA}
\end{frame} 

\begin{frame}{Wzmacniacz operacyjny}
	\textbf{Wzmacniacz odwracający}\\
	\url{https://www.falstad.com/circuit/circuitjs.html?ctz=CQAgjCAMB0l3BWcMBMcUHYMGZIA4UA2ATmIxAXKQUgoFMBaMMAKADcRjCQUFuvwYFFHAgALLSS1p0BCwDmnbr37dcYkZBYAnJYOFhiBoSLDwd4I+DEbs2bmBsjccFgHcQdh06+XhWj0NjAytHDQC-aw0g8AxuCIEVSKSImLQ8ZL4oFgBDSJjfGIywJCYkM3g4EAYYSrjCLGxiBDwcIUpw5EqWAHlPe0iJWhjpFgB7EUJOiWIMqVheEX9PceQQKZEZjLQoWHgyQj5FiFYgA}
\end{frame} 

\begin{frame}{Wzmacniacz operacyjny}
	\textbf{Wzmacniacz nieodwracający}\\
	\url{https://www.falstad.com/circuit/circuitjs.html?ctz=CQAgjCAMB0l3BWcMBMcUHYMGZIA4UA2ATmIxAXKQUgoFMBaMMAKADcRjCQUFuvwYFFHAgALLSS1p0BCwDmnbr37dcYkZBYB3cMWFo8eg3ygsAhsavZs3MPs7gkTJGHjwQDbNDwIUYmkIJWzgyZ29fFUIUMGw8DGJcYmR3HRAbOwcMnlMtACcrFXTbHO5aNDgWAoEi+xMylK1dAWZhOsFhJqtW7owylgB5YszhCVp26RYAexEgkQliIylYeDJCPl4RTvTpxznaBaM0KBXQvo3XcBYgA}
\end{frame} 

\begin{frame}{Wzmacniacz operacyjny}
	\textbf{Wtórnik}\\
	\url{https://www.falstad.com/circuit/circuitjs.html?ctz=CQAgjCAMB0l3BWcMBMcUHYMGZIA4UA2ATmIxAXKQUgoFMBaMMAKADcRjCQUUAWTtzy1aEPrSQio0BCwDmgnv0WYUUKCwDu4YmrAZuYXUoGQWAQx17CtbGnA3O4JEyRh48EH2h4+GBGgIxHhcCEZq3r7+JJAYfPrMBMgeWiB2oo7pJhraXCDCVvkiqeFFhWB8piXGvAJZtRoA8mm+DrR8dm3qZtpZYJmt-cUATi0CQ2PZovDykw3YrardLAD26oSmXpDBFNIeZIQIhCiSPOrYq-kgG+riO5Kw8AdHJ+DgLEA}
\end{frame} 

\begin{frame}{Wzmacniacz operacyjny}
	\textbf{Układ odejmujący}\\
	\url{https://www.falstad.com/circuit/circuitjs.html?ctz=CQAgjCAMB0l3BWcMBMcUHYMGZIA4UA2ATmIxAXKQUgoFMBaMMAKAEMQU89xiUQALNn5g+IHmCRMkYePBAMU0BChXE8GQgLACuebFKUqEuDYWzFd3DDLmQWAd07dwYfnvCaojwcN78hETdvACdnCWDA135aWTgWMNEg9xdmGOR4sI80HmyEQigM+wBzcM58stwBQvsslxz-TnxCuPsAJUa08C5olpBCWiRaYeUWAHlfETEBYgKkmpYAN0aGt1zm2hE0cBqoUY61pp5Zo76Bil2YBBYABxXm+dUC4YXSx4r5qoWAe3F+6toAkg6hA5BgkFUhX4PGwLF+wS0hSBILBsHgZEI+UhsThIGI-yRwNyyj2chQAm0QMIkIgrCAA}
\end{frame} 


\begin{frame}{Wzmacniacz operacyjny}
	\textbf{Układ sumujący}\\
	\url{https://www.falstad.com/circuit/circuitjs.html?ctz=CQAgjCAMB0l3BWcA2aYDMZIA5nrzgEzYCcICA7OeZOQKYC0YYAUAIYjHac4gAs6Qj27cwSJkizw4IBjGmEcYPOmyQ+CEgmyYqtKfBYB3TtlHFTokkMjH+g4faFgLtgE6XwFgc9fhDJlychHyehBQ2LADmYRGe6OpQUHZBYNacfPrptoGZ4Okh+si0ORlZQoXBoe5lKPoUyHVJBiweYA1VtYQhzYYASuAdyvoWw808tEglUNAILAPtjd2hLtzLSRW0NkkwcwDyTo58JEu8JSwA9uAQyNX8kCTcU7CEU5zXIOiXXiC3SZmPajyFxvXyfb4YX53AGiEqweAkBoIZCvcDgFhAA}
\end{frame} 