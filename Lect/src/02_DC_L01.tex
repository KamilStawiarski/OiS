\sect{Wykład 1}

\begin{frame}{Wielkości fizyczne}
	\begin{block}{Ładunek elektryczny}
	Wielkość fizyczna charakteryzująca ciała podlegające działaniu pola elektromagnetycznego i będące źródłem takiego pola.\footnote{https://encyklopedia.pwn.pl/haslo/;3934793}.
	Jednostką ładunku jest kulomb (symbol $C$) i stanowi około $6.241*10^18$ ładunków pojedynczego elektronu.
	Na ogół ładunek oznacza się symbolem $q$.
	\end{block}
\end{frame}

\begin{frame}{Wielkości fizyczne}
	\begin{block}{Natężenie prądu}
	Określa tempo przepływu ładunku w czasu.
	Jednostką natężenia jest amper ($1A=\frac{1C}{1s}$).
	Na ogół przepływ prądu oznacza się za pomocą symbolu $i$, który można przedstawić jako: $i(t(=\frac{dq(t)}{dt}$ lub $q(t)=\int_{-\infty}^{t} i(\tau) d \tau$.
	Natężenie mierzy się w punkcie obwodu, przez który przepływa prąd.
	\end{block}
\end{frame}

\begin{frame}{Wielkości fizyczne}
	\begin{table}
		\centering
		\tagpdfsetup{table/header-rows={1}}
		\caption{Przykładowe wartości natężenia}
		\begin{tabular}{|c|c|}
			\hline
			Natężenie [A] & Zjawisko / urządzenie \\ \hline
			$\approx 10^5$ & Piorun \\ \hline
			$\approx 10^3$ & Duży silnik przemysłowy \\ \hline
			$\approx 10$ & Płyta grzewcza \\ \hline
			$\approx 2$ & Prąd ładowania telefonu \\ \hline
			$\approx 30*10^-3$ & Umowna granica prądu niebezpiecznego dla życia \\ \hline
			$\approx 10^-6$ & Prąd pobierany przez uśpiony mikrokontrolwer \\ \hline
			$\approx 10^-12$ & Prąd synaptyczny \\ \hline
		\end{tabular}
	\end{table}
\end{frame}

\begin{frame}{Wielkości fizyczne}
	\begin{block}{Napięcie elektryczne}
	Określa ilość pracy jaką należy wykonać aby przemieścić ładunek $q$ między dwoma punktami.
	Na ogół napięcie oznacza się symbolem $u$.
	Wyrażone w voltach ($V$).
	Napięcie mierzy się między dwoma punktami.
	\end{block}
\end{frame}

\begin{frame}{Wielkości fizyczne}
	\begin{table}
		\centering
		\tagpdfsetup{table/header-rows={1}}
		\caption{Przykładowe wartości napięcia}
		\begin{tabular}{|c|c|}
			\hline
			Napięcie [V] & Zjawisko / urządzenie \\ \hline
			$\approx 10^8$ & Piorun \\ \hline
			$\approx 10^6$ & Linie przesyłowe najwyższych napięć \\ \hline
			$\approx 10^4$ & Duże silniki przemysłowe \\ \hline
			$\approx 230$ & Napięcie sieciowe \\ \hline
			$\approx 20$ & Ładowarka laptopa \\ \hline
			$\approx 10^-3$ & Napięcie EKG \\ \hline
			$\approx 10^-5$ & Napięcie EEG \\ \hline
			$\approx 10^-6$ & Napięcie na wyjściu anteny odbiornika radiowego \\ \hline
		\end{tabular}
	\end{table}
\end{frame}

\begin{frame}{Prawo Ohma}
	\begin{block}{Prawo Ohma}
		Natężenie prądu ($i$) jest wprost proporcjonalne do przyłożonego do przewodnika napięcia ($u$). \\
	\end{block}
	Proporcjonalność ta jest uzależniona od \textbf{rezystancji}, która jest z kolei własnością konkretnego przewodnika.
\end{frame}

\begin{frame}{Wielkości fizyczne}
	\begin{block}{Rezystancja}
		Własność przewodnika określająca stosunek przyłożonego napięcia do przepływającego prądu. \\
		Rezystancję na ogół oznacza się symbolem $R$, jej jednostką jest om $\Omega$. \\
		Zgodnie z prawem Ohma: $u(t)=\frac{i(t)}{R}$. \\
		Zazwyczaj wykorzystuje się 3 formy niezależne od czasu: $R=\frac{u}{i}$; $i=\frac{u}{R}$; $u=iR$. \\
		Rezystancja jest własnością przewodnika, zależy od materiału wykonania, gabarytów itp.
	\end{block}
\end{frame}

\begin{frame}{Wielkości fizyczne}
	\begin{table}
		\centering
		\tagpdfsetup{table/header-rows={1}}
		\caption{Przykładowe wartości rezystancji}
		\begin{tabular}{|c|c|}
			\hline
			$\approx 10^{12}$ & Rezystancja powietrza / dobrych izolatorów w warunkach laboratoryjnych \\ \hline
			$\approx 10^9$ & Rezystancja tworzyw sztucznych (izolatory) \\ \hline
			$\approx 10^7$ & Rezystancja ciała człowieka (sucha skóra) \\ \hline
			$\approx 10^5$ & Rezystancja ciała człowieka (wilgotna skóra) \\ \hline
			$\approx 10^3$ & Słuchawki, czujniki, rezystory w elektronice cyfrowej \\ \hline
			$\approx 10^2$ & Mały rezystor w układach elektronicznych \\ \hline
			$\approx 1$ & Typowy głośnik niskoomowy / uzwojenie transformatora \\ \hline
			$\approx 10^{-3}$ & Rezystancja grubego przewodu miedzianego (kilka metrów) \\ \hline
		\end{tabular}
	\end{table}
\end{frame}

\begin{frame}{Rezystor}
	\begin{columns}
		\begin{column}{0.5\textwidth}
			\begin{block}{Rezystor}
				Jeden z podstawowych, pasywnych, elementów elektronicznych.
				Idealny rezystor cechuje się stałą, niezmienną rezystancją.
				W rzeczywistości rezystory charakteryzują się również maksymalną mocą, z jaką mogą pracować, stałością parametrów itp.
			\end{block}
		\end{column}
		\begin{column}{0.5\textwidth}
			\begin{figure}
				\includegraphics[scale=0.2, alt={Na zdjęciu widać szereg rezystorów o różnych mocach znamionowych, od bardzo małych oporników węglowych 1/8 W do większych, cylindrycznych elementów 2 W. Poniżej pokazane są większe rezystory ceramiczne o mocach 10 W i 20 W, przy linijkach z podziałką w centymetrach do porównania ich rozmiarów.}]{./imags/L1/Carbon_and_ceramic_resistors_of_different_power_ratings.jpg}	
				\caption{Przekładowe rezystory różnych mocy.}
			\end{figure}	
		\end{column}
	\end{columns}
\end{frame}


\begin{frame}{Energia i Moc w obwodach}
	\begin{block}{Energia}
		Jak wcześniej wspomniano, różnica potencjałów między punktami A i B wynosząca $1V$ oznacza, że przetransportowanie ładunku $q=1C$ z punktu A do B wymaga dostarczenie $1J$ energii.
		Oznacza to jednocześnie, że ten sam ładunek poruszający się ob punktu B do A odda energię $1J$.\\
		$E=u*q$
	\end{block}
	\begin{block}{Moc}
		Moc informuje o zmianie energii w czasie ($P(t)=\frac{dE(t)}{t}$).
		Zakładając, że zamiast pojedynczego ładunku między punktami A i B przepływa prąd elektryczny - do układu dostarczana jest moc.\\
		$P(t)=u(t)*\frac{d q(t)}{t}=u(t)*i(t)$ \\
		Często przyjmuje się prostszą postać: $P=u*i$, lub z wykorzystaniem prawa Ohma: $P=i^2 R = \frac{u^2}{R}$.
	\end{block}
\end{frame}

\begin{frame}{Przedrostki + notacja wykładnicza}
	\begin{table}
		\centering
		\tagpdfsetup{table/header-rows={1}}
		\caption{Przedrostki dziesiętne}
			\begin{tabular}{|c|c|c|}
			\hline
			Przedrostek & Symbol & Mnożnik \\ \hline
			piko  & p & $10^{-12}$ \\ \hline
			nano  & n & $10^{-9}$  \\ \hline
			mikro & $\mu$ & $10^{-6}$ \\ \hline
			mili  & m & $10^{-3}$ \\ \hline
			kilo  & k & $10^{3}$  \\ \hline
			mega  & M & $10^{6}$  \\ \hline
		\end{tabular}
	\end{table}
\end{frame}

\begin{frame}{Definicja obwodu elektrycznego}
	\begin{block}{Obwód elektryczny}
		Obwód elektryczny to zamknięty układ elementów przewodzących, w którym pod wpływem przyłożonego napięcia może płynąć prąd elektryczny.
	\end{block}
	Co tworzy obwód
	\begin{itemize}
		\item[\bt] Obwód zawiera co najmniej źródło zasilania oraz jeden odbiornik (np. żarówkę, rezystor), połączone przewodami w pętlę.
		\item[\bt] Jeśli ścieżka przewodzenia jest przerwana, obwód jest otwarty i prąd nie płynie; gdy połączenia tworzą pełną pętlę, obwód jest zamknięty i możliwy jest przepływ prądu.
	\end{itemize}
\end{frame}

\begin{frame}{Elementy w obwodzie}
	Poniżej wymieniono wyłącznie elementy występujące w obrębie niniejszego przedmiotu, w rzeczywistości ich lista jest znacznie dłuższa.
	\begin{itemize}
		\item[\bt] Rezystory, cewki, kondensatory
		\item[\bt] Źródła niezależne
		\item[\bt] Źródła zależne
		\item[\bt] Wzmacniacze operacyjne
    \end{itemize}
\end{frame}

\begin{frame}{Rezystor w obwodzie}
	\begin{columns}
		\begin{column}{0.7\textwidth}
			\begin{block}{Rezystor}
				Rezystor oznaczany jest za pomocą symbolu prostokąta z dwoma wyprowadzeniami na krótszych bokach.
				Prąd $i_R$ przepływający przez rezystor powoduje powstanie napięcia $u_R$.
				Napięcie na rezystorze zawsze strzałkowane jest przeciwnie do przepływu prądu.
				Na rezystorze moc może być jedynie wydzielana, rezystor nie jest w stanie dostarczyć mocy do obwodu.
			\end{block}
		\end{column}
		\begin{column}{0.3\textwidth}
			\begin{figure}
				\includegraphics[scale=0.15, alt={Na rysunku jest symbol rezystora z zaznaczonymi zwrotami prądu iR oraz napięcia uR na tym elemencie. 
				Strzałka prądu skierowana jest w dół przez rezystor, a strzałka napięcia w górę, co ilustruje przyjętą biegunowość przy zapisie równań obwodu.}]{./imags/L1/R.png}	
				\caption{Rezystor z przepływającym prądem i strzałką napięcia.}
			\end{figure}	
		\end{column}
	\end{columns}
\end{frame}

\begin{frame}{Rezystor w obwodzie - symulacja}
	\begin{block}{Symulacja zachowania rezystora}
		\url{https://www.falstad.com/circuit/circuitjs.html}
	\end{block}
\end{frame}


\begin{frame}{Źródło napięciowe}
	\begin{columns}
		\begin{column}{0.7\textwidth}
			\begin{block}{Źródło napięciowe}
				Idealne źródło napięciowe stanowi element na którym zawsze występuje napięcie $E$, niezależnie od przepływu prądu $i_E$.
				Idealne źródło napięciowe nie występuje w rzeczywistości, ale podobnie do niego zachowują się np. akumulatory, baterie, zasilacze.
				Strzałka wewnętrzna wskazuje zwrot napięcia i zawsze jest zgodna z strzałką napięcia $E$.
				Na ogół oznacza się że zwrot przepływu prądu $i_E$ jest zgodny ze strzałką napięcia.
				Źródło napięciowe może zarówno dostarczać, jak i odbierać moc w obwodzie.
			\end{block}
		\end{column}
		\begin{column}{0.3\textwidth}
			\begin{figure}
				\includegraphics[scale=0.15, alt={Na rysunku jest symbol źródła napięcia (okrąg) z zaznaczonym prądem iE płynącym pionowo w górę przez element. 
				Obok narysowano strzałkę z literą E, oznaczającą siłę elektromotoryczną lub napięcie źródła, ze zwrotem ku górze.}]{./imags/L1/E.png}	
				\caption{Idealne źródło napięciowe z przepływającym prądem i strzałką napięcia.}
			\end{figure}	
		\end{column}
	\end{columns}
\end{frame}

\begin{frame}{Źródło napięciowe - symulacja}
	\begin{block}{Symulacja zachowania źródła napięciowego}
		\url{https://www.falstad.com/circuit/circuitjs.html}
	\end{block}
\end{frame}

\begin{frame}{Źródło prądowe}
	\begin{columns}
		\begin{column}{0.7\textwidth}
			\begin{block}{Źródło prądowe}
				Idealne źródło prądowe stanowi element przez który zawsze przepływa prąd $J$, niezależnie od napięcia $u_J$ na nim.
				Idealne źródło prądowe nie występuje w rzeczywistości, przez swój charakter nie istnieją powszechnie znane elementy o podobnym zachowaniu, co sprawia że jest trudniejszy w zrozumieniu.
				Strzałka wewnętrzna wskazuje zwrot przepływu prądu $J$.
				Na ogół oznacza się że zwrot napięcia $u_J$ jest zgodny ze strzałką prądu.
				Źródło prądowe może zarówno dostarczać, jak i odbierać moc w obwodzie.
			\end{block}
		\end{column}
		\begin{column}{0.3\textwidth}
			\begin{figure}
				\includegraphics[scale=0.15, alt={Na rysunku przedstawiono symbol źródła prądowego (podwójny okrąg) z prądem J płynącym pionowo w górę przez element. 
				Obok znajduje się strzałka z oznaczeniem uJ, określająca przyjęty zwrot napięcia na źródle prądowym.}]{./imags/L1/J.png}	
				\caption{Idealne źródło prądowe z przepływającym prądem i strzałką napięcia.}
			\end{figure}	
		\end{column}
	\end{columns}
\end{frame}

\begin{frame}{Źródło prądowe - symulacja}
	\begin{block}{Symulacja zachowania źródła prądowego}
		\url{https://www.falstad.com/circuit/circuitjs.html}
	\end{block}
\end{frame}

\begin{frame}{Prądowe prawo Kirchoffa (PPK)}
	Prądowe prawo Kirchhoffa (PPK), zwane też pierwszym prawem Kirchhoffa, mówi, że w każdym węźle obwodu elektrycznego suma algebraiczna prądów jest równa zero. 
	Innymi słowy, suma natężeń prądów wpływających do węzła jest równa sumie natężeń prądów z niego wypływających, co wynika z zasady zachowania ładunku elektrycznego – ładunek nie może się w węźle ani gromadzić, ani znikać. 
	Prawo to zapisuje się zwykle jako $\sum i=0$, przy czym prądom wpływającym nadaje się znak dodatni, a wypływającym ujemny.
\end{frame}

\begin{frame}{Napięciowe prawi Kirchofa (NPK)}
	Napięciowe prawo Kirchhoffa (NPK), nazywane też drugim prawem Kirchhoffa, mówi, że w każdym zamkniętym oczku obwodu algebraiczna suma wszystkich napięć jest równa zeru. 
	Równoważnie można powiedzieć, że suma spadków napięć na elementach odbiorczych w oczku jest równa sumie sił elektromotorycznych źródeł w tym oczku. 
	Matematycznie zapisuje się to zwykle jako $\sum u=0$, przy czym napięcia zgodne z przyjętym kierunkiem obiegu oczka przyjmuje się z plusem, a przeciwne – z minusem.
\end{frame}

\begin{frame}{Zasada zachowania energii (zasada Tellgena)}
	Zasada zachowania energii mówi, że w układzie izolowanym całkowita energia nie może być ani stworzona, ani zniszczona - może jedynie zmieniać formę (np. z elektrycznej w cieplną).
	\begin{block}{Zasada Tellegena w obwodach}
		\begin{itemize}
			\item[\bt] W teorii obwodów odpowiednikiem tej zasady jest twierdzenie Tellegena: w dowolnym obwodzie skupionym suma mocy dostarczanych przez wszystkie elementy jest równa sumie mocy przez nie pobieranych, czyli suma algebraiczna mocy we wszystkich gałęziach jest równa zero.
			\item[\bt] Oznacza to, że energia dostarczana przez źródła napięcia i prądu w obwodzie w każdej chwili dokładnie równoważy energię zużywaną lub magazynowaną w elementach biernych (rezystorach, cewkach, kondensatorach), co jest bezpośrednim zastosowaniem ogólnej zasady zachowania energii do sieci elektrycznych.
		\end{itemize}
	\end{block}
\end{frame}

\begin{frame}{Definicje SLS}
	W elektrotechnice SLS oznacza "skupiony, liniowy, stacjonarny" obwód elektryczny - układ z elementami skupionymi, spełniającymi prawo Ohma i o parametrach niezmieniających się w czasie.
\end{frame}