\sect{Wykład 8}

\begin{frame}{Elementy liniowe inercyjne}
		\begin{itemize}
		\item[\bt] Element elektroniczny inercyjny to element, dla którego
płynący prąd chwilowy zależy nie tylko od chwilowego napięcia, ale także od przebiegu tego napięcia w przeszłości.
		\item[\bt] Elementy inercyjne mają zdolność gromadzenia energii.
		\item[\bt] Są nimi kondensatory i cewki indukcyjne.
	\end{itemize}
\end{frame} 


\begin{frame}{Kondensator liniowy}
	\begin{columns}
		\begin{column}{0.6\textwidth}
			\begin{itemize}
				\item[\bt] Element przechowujący energię w postaci pola elektrycznego.
				\item[\bt] W najprostszej postaci składa się z dwóch okładzin przedzielonych dielektrykiem.
				\item[\bt] Z perspektywy prądu stałego stanowi przerwę w obwodzie.
			\end{itemize}
		\end{column}
		\begin{column}{0.4\textwidth}
			\begin{figure}
				\includegraphics[scale=0.08, alt={Na obrazku jest symbol kondensatora z zaznaczonym kierunkiem przepływu prądu oraz napięcia.​
				Pośrodku widoczne są dwie poziome, równoległe kreski – to płytki kondensatora. Z dolnej płytki w dół wychodzi pionowa linia – to wyprowadzenie kondensatora w dół. Z górnej płytki w górę biegnie krótka pionowa linia zakończona trójkątną strzałką skierowaną w dół - ta strzałka oznacza kierunek prądu o nazwie i c, napis i c znajduje się nad strzałką po prawej stronie. Po lewej stronie symbolu, obok dolnej części kondensatora, jest pionowa strzałka skierowana w górę - obok niej, po lewej, znajduje się oznaczenie napięcia u c. Po prawej stronie kondensatora, mniej więcej na wysokości środka, widoczna jest duża litera C, oznaczająca pojemność kondensatora.​}]{./imags/L8/C.png}	
				\caption{Symbol kondensatora.}
			\end{figure}	
		\end{column}
	\end{columns}
\end{frame} 


\begin{frame}{Kondensator liniowy}
	\begin{figure}
		\includegraphics[scale=0.5, alt={Na zdjęciu widać różne rodzaje kondensatorów leżących na jasnym, gładkim podłożu.
		Po lewej stronie są dwa duże kondensatory elektrolityczne w metalowych cylindrycznych obudowach, z których wychodzą cienkie druty. Jeden jest ciemnoniebieski z białymi napisami, drugi srebrny, bardziej przypominający metalową rurkę. Powyżej i po prawej stronie leży kilka mniejszych, również cylindrycznych kondensatorów w niebieskich i czarnych obudowach, także z wyprowadzeniami z drutu.
		W prawej części zdjęcia znajdują się dwa elementy przypominające małe, metalowe mechanizmy z białymi i metalowymi częściami - to nastawne kondensatory, z widocznymi blaszkami, które można regulować. Na dole zdjęcia ułożony jest rząd małych kondensatorów ceramicznych i foliowych w różnych kolorach: niebieskim, pomarańczowym, zielonym i czarnym, każdy z dwoma prostymi drucikami skierowanymi w dół. Na niektórych widoczne są drobne nadruki z liczbami i literami oznaczającymi ich parametry.​}]{./imags/L8/Capacitors.jpeg}	
		\caption{Przykładowe kondensatory.}
	\end{figure}	
\end{frame} 

\begin{frame}{Kondensator liniowy}
	Równanie kondensatora:\\
	$q_C=C \cdot u_C$ \\
	\begin{itemize}
		\item[\bt] $q_C$ - ładunek kondensatora w kulombach $\left[C=A \cdot s\right]$
		\item[\bt] $C$ - pojemność kondensatora w faradach $\left[F=C/V \right]$
		\item[\bt] $u_C$ - spadek napięcia na kondensatorze w woltach $\left[ V\right]$
	\end{itemize}
\end{frame} 

\begin{frame}{Kondensator liniowy}
	$i_C(t) = \frac{d q_C (t)}{dt} = C \frac{d u_C(t)}{dt} $\\
	$u_C(t) = u_C(t_0) + \frac{1}{C} \int_{t_0}^t i_C(\tau) d \tau$\\
	$E_C(t_0; t)= \frac{C}{2} \left( u^2_C(t) - u^2_C(t_0) \right)$\\
	Symulacja 1:\\
	\url{https://www.falstad.com/circuit/circuitjs.html?ctz=CQAgjCAMB0l3BWcMBMcUHYMGZIA4UA2ATmIxExCQBZsqBTAWjDACgB3Cla8an7BCl49IHLjxQJCIAUMnTRAS3EUpKsHyjhWAYxmDh+uWsjgosSBBhw2AexkhCIkNUilzYKZPDQFFB9isQA} \\
	Symulacja 2:\\
	\url{https://www.falstad.com/circuit/circuitjs.html?ctz=CQAgjCAMB0l3BWcMBMcUHYMGZIA4UA2ATmIxAUgpABZsKBTAWjDACgBjEFFGkbbIW69uNKlQgw4k2JHYBlYX15UefMDT7iQAMwCGAGwDODKNzYA3cCjz9B12yrNUx4Z1GgI2Adwd2hav5QPn5OYDbgmsEATn4ayviRWuAhgQJC2GhBkKmu8fxZ+Tm+mRJRpaLibLEVThXpztDsQA} \\
\end{frame} 


\begin{frame}{Kondensator liniowy}
	Połączenie równoległe kondensatorów:\\
	$C_Z=\sum_{n=1}^N C_n$\\
	Połączenie szeregowe kondensatorów:\\
	$\frac{1}{C_Z}=\sum_{n=1}^N \frac{1}{C_n}$\\
\end{frame} 

\begin{frame}{Kondensator liniowy}
	\textbf{Jak to działa że przez przerwę w obwodzie płynie prąd?}\\
	Przez kondensator nie płynie prąd w taki sposób, jak przez zwykły przewód, mimo że między jego okładkami jest przerwa. Dzieje się tak dlatego, że kondensator gromadzi i oddaje ładunek na swoich dwóch metalowych okładkach oddzielonych izolatorem, a prąd w obwodzie jest związany ze zmianą tego ładunku w czasie.\\
	\textbf{Co to znaczy "przerwa"}\\
	W środku kondensatora jest izolator, więc elektrony nie przeskakują z jednej okładki na drugą - gdy napięcie jest stałe i nic się nie zmienia, prąd ustaje, jak przy zwykłej przerwie w obwodzie. \\
	Dlatego w stanie ustalonym dla prądu stałego kondensator zachowuje się jak przerwa: po naładowaniu prąd już nie płynie.\\
\end{frame} 

\begin{frame}{Kondensator liniowy}
	\textbf{Skąd więc bierze się prąd}\\
	Kiedy napięcie na kondensatorze zaczyna się zmieniać (np. w chwili podłączenia baterii albo przy prądzie zmiennym), na jednej okładce gromadzi się dodatkowy ładunek, a z drugiej tyle samo ładunku "ucieka" do reszty obwodu.\\
	Ta zmiana ilości ładunku powoduje przepływ prądu w przewodach doprowadzających, choć sam izolator między okładkami wciąż niczego nie przewodzi.\\
	\textbf{Jak to opisać prostymi słowami}\\
    Można wyobrazić sobie dwie metalowe płytki oddzielone cienką szybą: nie ma przejścia, ale jeśli na jedną płytkę doprowadzimy dodatni ładunek, to z drugiej płytki odepchniemy dodatni ładunek do obwodu - prąd w przewodach popłynie "dookoła", bez skakania przez szybę.\\
    Im szybciej zmienia się napięcie (szybciej ładujemy lub rozładowujemy kondensator), tym większy prąd pojawia się w obwodzie; jeśli napięcie przestaje się zmieniać, prąd spada do zera.\\
\end{frame} 

\begin{frame}{Kondensator liniowy}
    \textbf{Dlaczego przy prądzie zmiennym to działa ciągle}\\
    Przy prądzie przemiennym napięcie stale rośnie i maleje, więc kondensator jest cały czas naładowywany i rozładowywany, a prąd w obwodzie nieustannie płynie.\\
    Dla bardzo wolnych zmian (prawie prąd stały) kondensator "prawie" odcina prąd, a dla szybkich zmian (wysoka częstotliwość) zachowuje się prawie jak zwykły przewodnik.
\end{frame} 

\begin{frame}{Kondensator liniowy}
	\textbf{	Zastosowania}\\
	\begin{itemize}
		\item[\bt] Wygładzanie tętnień w zasilaczach.
		\item[\bt] Odsprzęganie i stabilizacja zasilania układów scalonych.
		\item[\bt] Filtry sygnałowe (dolno-, górno- i pasmowo-przepustowe).
		\item[\bt] Sprzęganie stopni wzmacniaczy (przekazywanie składowej zmiennej sygnału).
		\item[\bt] Krótkotrwałe magazynowanie energii i podtrzymanie zasilania.
		\item[\bt] Superkondensatory w systemach odzyskiwania energii i pojazdach elektrycznych.
		\item[\bt] Kondensatory rozruchowe i pracy w silnikach jednofazowych.
		\item[\bt] Kompensacja mocy biernej w sieciach elektroenergetycznych.
		\item[\bt] Obwody rezonansowe i generatory częstotliwości.
		\item[\bt] Czujniki pojemnościowe (dotykowe, zbliżeniowe, pomiar poziomu cieczy).
	\end{itemize}
\end{frame} 


\begin{frame}{Induktor liniowy}
	\begin{columns}
		\begin{column}{0.6\textwidth}
			\begin{itemize}
				\item[\bt] Element przechowujący energię w postaci pola magnetycznego.
				\item[\bt] W najprostszej postaci składa się z spirali uformowanej z przewodnika.
				\item[\bt] Z perspektywy prądu stałego stanowi zwarcie w obwodzie.
			\end{itemize}
		\end{column}
		\begin{column}{0.4\textwidth}
			\begin{figure}
				\includegraphics[scale=0.08, alt={Na obrazku jest symbol cewki indukcyjnej z zaznaczonym kierunkiem prądu oraz napięcia.
				Pośrodku znajduje się pionowa cewka narysowana jako kilka poziomych pętli, jedna pod drugą, połączonych w pionie. Z dołu cewki wychodzi pionowa linia w dół - to dolne wyprowadzenie cewki. U góry cewki biegnie krótki odcinek linii zakończony trójkątną strzałką skierowaną w dół; nad tą strzałką, po prawej stronie, umieszczony jest symbol prądu i L (i indeks L).
				Po lewej stronie cewki, na wysokości jej środka, narysowana jest pionowa strzałka skierowana w górę, oznaczająca kierunek napięcia na cewce. Obok tej strzałki, po lewej, znajduje się oznaczenie napięcia u L (u indeks L). Po prawej stronie, na wysokości środka cewki, widnieje duża litera L, oznaczająca indukcyjność cewki.​}]{./imags/L8/L.png}	
				\caption{Symbol cewki.}
			\end{figure}	
		\end{column}
	\end{columns}
\end{frame} 

\begin{frame}{Induktor liniowy}
	\begin{figure}
		\includegraphics[scale=0.15, alt={Na zdjęciu widać kilka cewek indukcyjnych w różnych wykonaniach, ułożonych na jasnym tle.
		Na górze znajduje się niebieski przewód owinięty dwukrotnie wokół ciemnego, okrągłego pierścienia z ferrytu, przypominającego mały, gruby pierścień lub obrączkę. Po prawej stronie tego przewodu widać odsłonięty, miedziany koniec żyły.
		Na dole, od lewej, stoi pionowo mały czarny walec, ciasno owinięty miedzianym drutem - końce drutu są wyprowadzone w dół jako dwa sztywne srebrne wyprowadzenia. Obok niego znajduje się mała, okrągła cewka na żółtym rdzeniu pierścieniowym, gęsto nawinięta cienkim miedzianym drutem. Po prawej stronie widoczny jest mały zielony element w kształcie zaokrąglonego walca z dwoma krótkimi, lekko powyginanymi drucikami wyprowadzonymi w dół.​}]{./imags/L8/Inductors.jpeg}	
		\caption{Przykładowe cewki.}
	\end{figure}	
\end{frame} 

\begin{frame}{Induktor liniowy}
	Równanie cewki:\\
	$\phi_L=L \cdot i_L$ \\
	\begin{itemize}
		\item[\bt] $\phi_L$ - strumień indukcji magnetycznej w weberach $\left[Wb=V \cdot s\right]$
		\item[\bt] $L$ - indukcyjność magnetyczna w henrach $\left[F=Wb/A \right]$
		\item[\bt] $i_L$ - prąd induktora w amperach $\left[ A\right]$
	\end{itemize}
\end{frame} 

\begin{frame}{Induktor liniowy}
	$u_L(t) = \frac{d \phi_L (t)}{dt} = L \frac{d i_L(t)}{dt} $\\
	$i_L(t) = i_L(t_0) + \frac{1}{L} \int_{t_0}^t u_L(\tau) d \tau$\\
	$E_L(t_0; t)= \frac{L}{2} \left( i^2_L(t) - i^2_L(t_0) \right)$\\
	Symulacja 1:\\
	\url{https://www.falstad.com/circuit/circuitjs.html?ctz=CQAgjCAMB0l3BWcMBMcUHYMGZIA4UA2ATmIxExCQBZsqBTAWjDACgB3Cla8an7BCl49IHLjxQJCIAUMnTRAN3EUpKsHyhbqkcFt0wErADYzBws3LW6IB1gHsZIQiJA7SVKNFsUn2VkA} \\
	Symulacja 2:\\
	\url{https://www.falstad.com/circuit/circuitjs.html?ctz=CQAgjCAMB0l3BWcMBMcUHYMGZIA4UA2ATmIxAUgpABZsKBTAWjDACgBlEFFG7mqjz5gafKlQBmAQwA2AZwZRubAG7gUeENmyF1m3uKUDwS8dARsA7nq07uvW7shWbBmyLFsATu9Hd84H5U7NZCjlpo4c6hxh4RwUEu2JFxyYLGzj5p-FTZ2k7IbDL2fPklOUoQTDBgKNii9RhghNgIzXiE5M5AA} \\
\end{frame} 

\begin{frame}{Induktor liniowy}
	Połączenie równoległe cewek:\\
	$\frac{1}{L_Z}=\sum_{n=1}^N \frac{1}{L_n}$\\
	Połączenie szeregowe cewek:\\
	$L_Z=\sum_{n=1}^N L_n$\\
\end{frame} 

\begin{frame}{Induktor liniowy}
	\textbf{Jak to działa że na zwiniętym kawałku przewodnika pojawia się napięcie?}\\
	Na zwiniętym kawałku przewodnika pojawia się napięcie dlatego, że zmieniający się prąd wytwarza zmienne pole magnetyczne, a to z kolei "przecina" zwoje cewki i wymusza w nich napięcie – to zjawisko nazywa się indukcją elektromagnetyczną.\\
	Gdy przez cewkę zaczyna płynąć prąd lub jego wartość się zmienia, wokół drutu powstaje i zmienia się pole magnetyczne.\\
	Zmiana tego pola powoduje powstanie siły elektromotorycznej (napięcia) w samym przewodniku; napięcie ma taki kierunek, aby sprzeciwiać się zmianie prądu (prawo Lenza).\\
	Im więcej zwojów ma cewka i im szybciej zmienia się prąd, tym większe napięcie się na niej pojawia.
\end{frame} 


\begin{frame}{Induktor liniowy}
	\textbf{Zastosowania}\
	\begin{itemize}
		\item[\bt] Filtry sygnałowe (dolno-, górno- i pasmowo-przepustowe).
		\item[\bt] Dławiki w zasilaczach do tłumienia zakłóceń i tętnień prądu.
		\item[\bt] Obwody rezonansowe LC w generatorach i odbiornikach radiowych.
		\item[\bt] Transformatory i przekładniki prądowe oraz napięciowe.
		\item[\bt] Cewki w przetwornicach impulsowych (magazynowanie energii w polu magnetycznym).
		\item[\bt] Elektromagnesy, przekaźniki i cewki styczników.
		\item[\bt] Zapłonowe cewki wysokiego napięcia w silnikach spalinowych.
		\item[\bt] Cewki odchylające i korekcyjne w urządzeniach wizyjnych.
		\item[\bt] Cewki w układach filtrów przeciwzakłóceniowych EMI.
		\item[\bt] Indukcyjne czujniki zbliżeniowe i detektory metalu.
	\end{itemize}
\end{frame} 


\begin{frame}{Elementy liniowe inercyjne}
	\begin{columns}
		\begin{column}{0.5\textwidth}
			\textbf{	Kondensator}
			\begin{itemize}
				\item[\bt] Rozwarcie dla prądu stałego
				\item[\bt] Prąd ~ pochodna napięcia
				\item[\bt] Napięcie ~ całka prądu
			\end{itemize}
			$i_C(t)=C \frac{d u_C(t)}{dt}$
		\end{column}
		\begin{column}{0.5\textwidth}
			\textbf{	Cewka}
			\begin{itemize}
				\item[\bt] Zwarcie dla prądu stałego
				\item[\bt] Prąd ~ całka napięcia
				\item[\bt] Napięcie ~ pochodna prądu
			\end{itemize}
			$u_L(t)=L \frac{d i_L(t)}{dt}$
		\end{column}
	\end{columns}
\end{frame} 


\begin{frame}{Indukcyjności sprzężona}
	\begin{columns}
		\begin{column}{0.5\textwidth}
			\begin{itemize}
				\item[\bt] $M$ - wzajemna indukcyjność magnetyczna
				\item[\bt] $k$ - stała konstrukcyjna
			\end{itemize}
			$u_1(t)=L_1 \frac{d i_1(t)}{dt} + M \frac{d i_2(t)}{dt}$\\
			$u_2(t)=M \frac{d i_1(t)}{dt} + L_2 \frac{d i_2(t)}{dt}$\\
			$M=k\sqrt{L_1 L_2}$\\
		\end{column}
		\begin{column}{0.5\textwidth}
			\begin{figure}
				\includegraphics[scale=0.08, alt={Na rysunku jest symbol transformatora przedstawiony jako dwie sprzężone cewki z zaznaczonymi kierunkami prądów i napięć po stronie pierwotnej i wtórnej.
				Po lewej stronie znajduje się pierwsza cewka opisana jako L1 - narysowana jest pionowo, w postaci kilku zakrzywionych łuków, tak jak standardowy symbol cewki. Po jej lewej stronie, wzdłuż pionowego odcinka przewodu, widnieje strzałka napięcia skierowana do góry z oznaczeniem U1 poniżej. Nad górnym końcem cewki poprowadzony jest poziomy przewód w lewo, na którym znajduje się strzałka skierowana w prawo, opisująca prąd I1.
				Po prawej stronie pierwszej cewki, w niewielkiej odległości, jest druga, podobna cewka opisana jako L2, ustawiona równolegle do L1. Obok górnych końców obu cewek, po wewnętrznych stronach, znajdują się małe, czarne kropki oznaczające zgodność biegunowości uzwojeń. Po prawej stronie drugiej cewki narysowana jest pionowa strzałka napięcia skierowana w górę z oznaczeniem U2, a nad jej górnym końcem, na poziomym przewodzie, widnieje strzałka prądu I2 skierowana w lewo.​}]{./imags/L8/T.png}
				\caption{Symbol transformatora.}
			\end{figure}	
		\end{column}
	\end{columns}
\end{frame} 


\begin{frame}{Indukcyjności sprzężona}
	\begin{figure}
		\includegraphics[scale=0.25, alt={Na zdjęciu widać mały transformator sieciowy stojący na metalowej podstawie, ustawiony na jasnym tle w delikatny wzorek.
		Dolna część transformatora to szeroka, prostokątna metalowa podstawa z otworami montażowymi w kształcie podłużnych owalnych szczelin, przez które można go przykręcić do obudowy. Na tej podstawie osadzony jest pakiet blaszanych płytek tworzących rdzeń transformatora – wygląda jak gruby, ciemny blok złożony z wielu cienkich warstw metalu.
		Na przedniej stronie, w dolnej części, widoczna jest owinięta uzwojeniami część transformatora, częściowo zalana żółtawą masą izolacyjną, która tworzy wypukłe "placki" i kapki przy krawędziach. Nad nią znajduje się listwa zaciskowa z kilkoma śrubowymi zaciskami, do których można podłączać przewody; w każdym z zacisków widać końcówki drutów uzwojeń.
		Na górze transformatora zamocowana jest jasna listwa z kilkoma zaciskami i małym czarnym elementem (np. bezpiecznikiem lub przełącznikiem), a z dwóch zacisków wychodzi pętla czarnego przewodu. Na samej górze rdzenia przyklejona jest prostokątna etykieta z danymi producenta i parametrami transformatora, ale napisy są na zdjęciu zbyt małe, aby je wyraźnie odczytać.​}]{./imags/L8/SmallTransformer.jpeg}	
		\caption{Przykładowy transformator.}
	\end{figure}	
\end{frame} 

\begin{frame}{Indukcyjności sprzężona}
	\textbf{Zastosowania}\
	\begin{itemize}
		\item[\bt] Dopasowanie poziomów napięcia między siecią energetyczną a urządzeniami odbiorczymi.
		\item[\bt] Separacja galwaniczna obwodów dla bezpieczeństwa i redukcji zakłóceń.
		\item[\bt] Transformatory sieciowe w zasilaczach urządzeń elektronicznych.
		\item[\bt] Transformatory mocy w liniach przesyłowych wysokiego i niskiego napięcia.
		\item[\bt] Przekładniki prądowe i napięciowe do pomiarów i zabezpieczeń.
		\item[\bt] Transformatory impulsowe w przetwornicach z izolacją galwaniczną.
		\item[\bt] Transformatory audio do dopasowania impedancji i separacji sygnału.
		\item[\bt] Transformatory w układach lampowych oraz wzmacniaczach mocy.
		\item[\bt] Transformatory w zasilaczach bezprzerwowych UPS i systemach zasilania awaryjnego.
		\item[\bt] Miniaturowe transformatory w układach komunikacyjnych (np. Ethernet, interfejsy telekomunikacyjne).
	\end{itemize}
\end{frame} 
